\documentclass{article}
\usepackage{amsmath,amsthm,amsfonts,amssymb,tikz,scalerel,mathrsfs,marvosym,titlesec,leftidx}

%for diagrams
\usepackage[all]{xy}

\usepackage{fancyhdr}
\usepackage{blkarray}
\usepackage{calrsfs}



%for referencing accross documents
\usepackage{xr-hyper}

\externaldocument[probability-]{probability}



%change the default font to bodoni
\usepackage[default]{gfsbodoni}
\usepackage[T1]{fontenc}

% Declare calligraphic font:
\DeclareMathAlphabet{\pr}{OMS}{zplm}{m}{n}


%start at section 0 for the overview: 
\setcounter{section}{-1}


%environments
%
\theoremstyle{plain}
\newtheorem{corollary}{Corollary}[section]
\newtheorem{convention}[corollary]{Convention}
\newtheorem{example}[corollary]{Example}
\newtheorem*{theorem*}{theorem}
\newtheorem{theorem}[corollary]{Theorem}
\newtheorem{lemma}[corollary]{Lemma}
\newtheorem*{lemma*}{Lemma}
\newtheorem{maintheorem}{Theorem}
%reset maintheorem counter to use the corresponding letter
\renewcommand{\themaintheorem}{\Alph{maintheorem}}
\newtheorem{remark}[corollary]{Remark}
\newtheorem{proposition}[corollary]{Proposition}


\theoremstyle{definition}
\newtheorem{definition}[corollary]{Definition}
\newtheorem{definition*}{Definition}
\newtheorem{notation}{Notation}

%macros
%
\newcommand{\A}{\mathcal{A}}
\DeclareMathOperator{\avg}{avg}
\newcommand{\abs}[1]{\vert #1 \vert }
\DeclareMathOperator{\ad}{ad}
\DeclareMathOperator{\Aff}{aff}
\DeclareMathOperator{\arginf}{arginf}
\DeclareMathOperator{\argmin}{argmin}
\DeclareMathOperator{\argsup}{argsup}
\DeclareMathOperator{\argmax}{argmax}
\DeclareMathOperator{\Aut}{Aut}
\renewcommand{\b}{\bullet}
\newcommand{\bl}[2]{\left\langle #1\,,#2\, \right\rangle}
\newcommand{\bp}{\mathbf{\Pi}}
\def\BVm{\operatorname{BV}_-}
\DeclareMathOperator{\BiMod}{BiMod}
\DeclareMathOperator{\bimod}{bimod}
\DeclareMathOperator{\card}{card}
\newcommand{\C}{{\mathcal{C}}}
\DeclareMathOperator{\CC}{CC}
\def\cd{\operatorname {cd}}
\DeclareMathOperator{\ch}{ch}
\DeclareMathOperator{\CH}{CH}
\DeclareMathOperator{\Conv}{Conv}
\DeclareMathOperator{\co}{co}
\DeclareMathOperator{\coh}{coh}
\DeclareMathOperator{\cone}{cone}
\DeclareMathOperator{\coker}{coker}
\renewcommand{\d}[1]{\mathbb{#1}}
\newcommand{\D}{{\mathscr{D}}}
\DeclareMathOperator{\der}{R}
\DeclareMathOperator{\Def}{Def}
\DeclareMathOperator{\Dir}{Dir}
\newcommand{\define}{\stackrel{\operatorname{def}}{=}}
\let\div\relax %unassign \div
\DeclareMathOperator{\div}{div}
\DeclareMathOperator{\divu}{div}
\newcommand{\draw}[1]{\widetilde{\textrm{S}} #1}
\newcommand{\ds}{\oplus}
\newcommand{\DS}{\bigoplus}
\newcommand{\epi}{\xymatrix{{}\ar@{->>}[r]&{}}}
\DeclareMathOperator{\End}{End}
\DeclareMathOperator{\Entropy}{Entropy}
\DeclareMathOperator{\Ext}{Ext}
\newcommand{\f}[1]{\mathfrak{#1}}
\DeclareMathOperator{\fd}{fd}
\newcommand{\fC}{\f{C}}
\newcommand{\fD}{\f{D}}
\newcommand{\floor}[1]{\left\lfloor #1 \right\rfloor}
\newcommand{\fun}{\mapsto}
\newcommand{\norm}[1]{\vert \vert #1 \vert \vert}
\DeclareMathOperator{\FRel}{FRel}
\newcommand{\dgg}{{\f{g}^\bullet}}
\newcommand{\g}{\f{g}}
\DeclareMathOperator{\Gd}{Gd}
\DeclareMathOperator{\ev}{ev}
\DeclareMathOperator{\gldim}{gl.dim}
\let\H\relax %unassign \H
\DeclareMathOperator{\H}{H}
\DeclareMathOperator{\HH}{HH}
\DeclareMathOperator{\HC}{HC}
\DeclareMathOperator{\Hom}{Hom}
\DeclareMathOperator{\Id}{Id}
\DeclareMathOperator{\ID3}{ID_3}
\DeclareMathOperator{\im}{im}
\newcommand{\iso}{\stackrel{\sim}{\longrightarrow}}
\DeclareMathOperator{\Jac}{Jac}
\renewcommand{\k}{\Bbbk}
\DeclareMathOperator{\kNN}{kNN}
\newcommand{\K}{\mathbb{K}}
\DeclareMathOperator{\length}{length}
\DeclareMathOperator{\lrk}{\l.rk}
\newcommand{\leftperp}[1]{\leftidx{^\perp}{#1}}
\DeclareMathOperator{\LS}{LS}
\newcommand{\m}{\f{m}}
\DeclareMathOperator{\Mat}{Mat}
\DeclareMathOperator{\MCM}{MCM}
\DeclareMathOperator{\MC}{MC}
\DeclareMathOperator{\MCcal}{\mathcal{MC}}
\DeclareMathOperator{\MLE}{MLE}
\DeclareMathOperator{\Mod}{Mod}
\DeclareMathOperator{\Mor}{Mor}
\def\mod{\operatorname{mod}}
\newcommand{\mor}{\longrightarrow}
\newcommand{\T}{\mathscr{T}}
\renewcommand{\O}{\mathcal{O}}
\DeclareMathOperator{\NS}{NS}
\DeclareMathOperator{\Num}{Num}
\newcommand{\p}{\mathfrak{p}}
\DeclareMathOperator{\per}{per}
\DeclareMathOperator{\Ob}{Ob}
\DeclareMathOperator{\Perf}{Perf}
\DeclareMathOperator{\Pic}{Pic}
\DeclareMathOperator{\Proj}{Proj}
\DeclareMathOperator{\poly}{poly}
\DeclareMathOperator{\pred}{pred}
\DeclareMathOperator{\Qcoh}{Qcoh}
\DeclareMathOperator{\QGr}{QGr}
\DeclareMathOperator{\Rel}{Rel}
\DeclareMathOperator{\RHom}{RHom}
\renewcommand{\r}[1]{\mathcal{#1}}
\DeclareMathOperator{\rk}{rk}
\DeclareMathOperator{\rrk}{r.rk}
\DeclareMathOperator{\rsample}{RS}
\def\locExt{\mathscr{E}\mathit{xt}}
\newcommand{\sample}[1]{\textrm{S} #1}
\DeclareMathOperator{\SL}{SL}
\DeclareMathOperator{\RS}{RS}
\DeclareMathOperator{\SPS}{SPS}
\DeclareMathOperator{\sing}{sing}
\DeclareMathOperator{\sep}{sep}
\DeclareMathOperator{\Test}{Test}
\DeclareMathOperator{\test}{test}
\DeclareMathOperator{\Train}{Train}
\DeclareMathOperator{\train}{train}
\DeclareMathOperator{\true}{true}
\DeclareMathOperator{\vectorspan}{span}
\DeclareMathOperator{\Spec}{Spec}
\DeclareMathOperator{\Supp}{Supp}
\DeclareMathOperator{\SVM}{SVM}
\DeclareMathOperator{\SupVec}{SupVec}
\DeclareMathOperator{\Sym}{Sym}
\def\locHom{\operatorname {\mathscr{H}\mathit{om}}}
\DeclareMathOperator{\Tails}{Tails}
\def\ltr{\overset{L}{\tr}}
\renewcommand{\c}[1]{\mathcal{#1}}
\newcommand{\mono}{\xymatrix{{}\ar@{^{(}->}[r]&{}}}\DeclareMathOperator{\num}{num}
\DeclareMathOperator{\rad}{rad}
\DeclareMathOperator{\Tr}{Tr}
\DeclareMathOperator{\Tor}{Tor}
\DeclareMathOperator{\Tors}{Tors}
\newcommand{\tr}{\otimes}
\newcommand{\trps}[1]{ \leftidx{^{\operatorname{tr}}}{#1}}
\def\uRHom{\operatorname {R\mathcal{H}\mathit{om}}}
\DeclareMathOperator{\Var}{Var}
\newcommand{\w}{{\tt{w}}}
\newcommand{\Q}{\mathbb{Q}}
\DeclareMathOperator{\qis}{qis}
\DeclareMathOperator{\Qvr}{\mathcal{Q}}
\newcommand{\Z}{\mathbb{Z}}


%add dots between section and page in toc
\usepackage{tocloft}
\renewcommand{\cftsecleader}{\cftdotfill{\cftdotsep}}


%imake nterline a little larger
\linespread{1.28}

%change the dimensions of the margin
\usepackage{geometry}
 \geometry{
 a4paper,
 total={210mm,297mm},
 left=23mm,
 right=23mm,
 top=20mm,
 bottom=20mm,
 }
 
%changeqedsymbol
\renewcommand{\qedsymbol}{\CrossedBox}

%make the (sub)subsections run into the text, the 0.25 is the distance between the numbering and title
\titleformat{\subsection}[runin]{\normalfont\Large\bfseries}{\thesubsection.}{0.25em}{}
\titleformat{\subsubsection}[runin]{\normalfont\normalsize\bfseries}{\thesubsubsection.}{0.25em}{}

% remove indentation:
\setlength\parindent{0pt}

\begin{document}
\title{LINEAR LEARNERS}

\maketitle	


\noindent\hrulefill
\tableofcontents
\noindent\hrulefill

\phantomsection
\label{section-phantom}

\section{Overview}

We begin our study of learners with one of the most ubiquitous learning algorithms: \emph{linear regression}.\\ in this context, we endow the  label space $\f{y}$ with the structure of a finite-dimensional inner product space. $\f{y}$ is thus equipped with a norm in particular. Before we continue our study of linear regression we recall the following standard consructions of inner product spaces:

\begin{lemma}\label{lem:innerproducts}
Let $V$ and $W$ be inner product spaces with orthonormal bases $\big(v_1,\ldots v_m\big) \in V$ and $\big(w_1,\ldots , w_n\big) \subset W$. Then
\begin{enumerate}
\item $V^\ast \define \Hom_\d{R}(V,\d{R})$ is an inner product space with $\bl{f}{g}\define \bl{v}{v'}$ if $\bl{v}{-}=f$ and  $\bl{-}{v'}=g$
and orthonormal basis given by the maps $\bigg( \bl{v_1}{-}, \ldots, \bl{v_m}{-} \bigg)$
\item $V\times W$ is an inner product space with $\bl{(v,w)}{(v',w')}\define \bl{v}{w}+\bl{v'}{w'}$ and orthonormal basis $\bigg((v_i,w_j)\bigg)_{i,j}$
\item $V\tr W$ is an inner product space with $\bl{v\tr w}{v'\tr w}\define \bl{v}{w}\cdot \bl{v'}{w'}$ and orthonormal basis $\bigg((v_i\tr w_j)\bigg)_{i,j}$	
\item $\Hom(V,W)$ is an inner product space with an orthonormal basis given by maps $e_{i,j}$ defined as
\begin{equation}
e_{i,j}(v_k)=
\begin{cases}
0, & \text{if}\ k \neq i  \\
w_j, & \text{otherwise}
\end{cases}
\end{equation}
\end{enumerate}
\end{lemma}


\begin{proof}
Items (1) through (3) consist of routine calculations. To show item (4), recall that the function
\[
\iota: V^*\tr W \mor \Hom_{\d{R}}(V,W)
\]
which assigns to $f\tr w$ the linear map $\iota_{f\tr w}(v)\define f(v)\cdot w$ is an isomorphism. Now, items $(1)$ and $(3)$ together imply that $V^*\tr W$ is indeed an inner product space with  orthonormal basis $\big\{\bl{v_i}{-}\tr w_j\big\}_{i,j}$. Now simply note that the map associated to $\bl{v_i}{-}\tr w_j$ under the function $\iota$ is exactly $e_{i,j}$
\end{proof}	
	
Returning to our discussion above, we consider a finite-dimensional inner product space $\f{y}$ of labels and $\f{X}$ any set of features. Now, for any finite dataset $\Delta\subset \f{X}\times \f{y}$, the space $\f{y}^\Delta$ in turn carries an inner product by Lemma \ref{lem:innerproducts}, so that for any map $f \in \f{y}^\f{X}$, we can define a cost as follows:
\[
c(\Delta,f)=\vert \vert y-f(x)\vert\vert_{\f{y}^\Delta}\define\sqrt{\sum_{\Delta} \vert \vert y-f(x)\vert \vert^2}
\]

The main result of this chapter will be to show that this cost indeed describes a learner under the right conditions. To this end, we will make the following definition:
\begin{definition}
Let $\f{H}\subset \f{y}^\f{X}$ and $\Delta \subset \f{X}\times \f{y}$. Then we say that $\Delta$ separates $\f{H}$ if for any $f,g \in \f{H}$:
\[
f\arrowvert_{\Delta_{\f{X}}}=g\arrowvert_{{\Delta}_\f{X}}\implies f=g
\]
\end{definition}
The main result of this chapter is: 
\begin{theorem}\label{thm:linearlearner}
Let $\f{y}$ be a finite-dimensional inner product space, $\f{X}$ any set and let $\f{H}\subset \f{y}^\f{X}$ be a finite-dimensional subspace of $\f{y}^\f{X}$.  Assume that  any $\Delta \in \f{D}$ separates $\f{H}$ and let $c(\Delta,f)= \vert\vert y-f(x)\vert \vert_{\f{y}^\Delta}$. Then $(\f{X},\f{y},\f{D},\f{H},c)$ defines a sharp learner (as in Definition \ref{def:sharp})
\end{theorem}

\noindent After having reviewed the necessary linear algebraic, we now have the ingredients to prove the promised main theorem of logistic regression, which we recall below for the reader's convenience: 
\begin{theorem}[linear regression]\label{thm:linearregression}
Let $\f{y}$ be a finite dimensional inner product space,  $\f{X}$ any set. Let $\f{H}\subset \f{y}^\f{X}$ be a finite dimensional subspace and \[
\f{D}=\bigg\{ \Delta \subset \f{X}\times \f{y}\, \, \bigg\vert \vert  \Delta \vert \neq \infty \textrm{ and } \Delta \textrm{ separates }\f{H} \bigg\}.
\]
and 
\[c: \f{D}\times \f{H}\mor \f{R}: (\Delta,f)\fun \vert \vert (y-f(x))_{(x,y)\in \Delta}\vert \vert_{\f{y}^\Delta}
\]
Then $\big(\f{X},\f{y},\f{D},\f{H},c\big)$ is a sharp learner.\\
\end{theorem}



\begin{proof}
Let $\Delta\in \f{D}$.\\
Since $\f{H}$ is finite dimensional, we can choose an inner product on it.  Moreover, the space $\f{y}^\Delta$ is canonically a finite dimensional inner product space by Lemma \ref{lem:innerproducts}\\
We now consider the map:
\[
\textrm{ev}_\Delta: \f{H}\mor \f{y}^\Delta: f \fun f(\Delta_\f{X})
\]
(where we used notation \ref{not:Delta}). It is easy to see that this map is linear. Moreover, the fact that  $\Delta$ separates $\f{H}$ is equivalent to $ev_\Delta$ being injective\\
WE note now that $\ev_\Delta$  allows us to rewrite the cost function $c(\Delta,f)$ as follows:
\[
c(\Delta,f)=\vert \vert \big(y-f(x)\big)_{(x,y)\in \Delta}\vert \vert_{\f{y}^\Delta}=\vert \vert \Delta_\f{y}-\textrm{ev}_\Delta(f)\vert \vert_{\f{y}^\Delta}
\]
It follows that $f$ that minimizes the cost $c(\Delta, f)$ iff $f$ minimizes the distance between $\Delta_{\f{y}}$ and  $\textrm{ev}_\Delta(f)$. By Lemma \ref{lem:projection}, this is in turn equivalent to requiring that $\textrm{ev}_\Delta(f)$ is the projection of the vector $\Delta_{\f{y}}$ onto the image of the map $\textrm{ev}_\Delta:\f{H}\mor \f{y}^\Delta$. We can now describe this image using Moore-Penrose inverses: indeed, let

\[
h_\Delta \define\textrm{ev}_\Delta^+\big(\Delta_{\f{y}}\big)
\]

 Then Lemma \ref{lem:proj=mppsinverse} implies that $\textrm{ev}_\Delta(f)$ is the projection of $\Delta_\f{y}$ onto the image of $\ev_\Delta^+$ if and only if $f$  lies in the affine subspace $h_\Delta+\ker(\textrm{ev}_\Delta)\subset \f{y}^\Delta$. We now recall that $\textrm{ev}_\Delta$ is in fact injective  so that finally $
f = h_\Delta
$
\end{proof}

As a corollary, we can show that the Euclidean learner introduced in Definition \ref{def:euclideanlearner} indeed is a learner provide an explicit formula for the learned hypothesis $h_\Delta$
\begin{corollary}\label{cor:euclideanlearners}
Let $\f{X}$ and $\f{y}$ be finite dimensional inner product spaces, 	
\[
\f{D} = \bigg\{ \Delta \subset \f{X}\times \f{y}\,\,\bigg\vert\,  \vert \Delta \vert \neq \infty , \textrm{ and span}(\Delta_{\f{X}}) = \f{X}\bigg\}
\text{,}\,\,\, \f{H}=\Hom_\d{R}(\f{X},\f{y})
\]
and 
\[
c:\f{D}\times \f{H}\mor \d{R}:(\Delta,f)\fun \vert \vert \big( y-f(x)\big)_{(x,y)\in \Delta} \vert \vert_{\f{y}^\Delta} 
\]
Then $(\f{X},\f{y},\f{D},\f{H},c)$ is a sharp linear learner. Moreover, 
\[
h_\Delta = \ev_\Delta^+\big(\Delta_{\f{y}}\big)=\bigg(\big(\ev_\Delta^*\circ \ev_\Delta\big)^{-1}\circ \ev_\Delta^*\bigg)(\Delta_\f{y})
\]
where $\ev_\Delta:\f{H}\mor \f{y}^\Delta: f \fun f(\Delta_{\f{y}})$
\end{corollary}

\begin{proof}
By Theorem \ref{thm:linearlearner}, we simply need to note that $\f{H}=\Hom_{\d{R}}(\f{X},\f{y}$ is finite dimensional and that each dataset $\Delta$ indeed separates $\f{H}$ since $\Delta_\f{X}$ spans the whole of $\f{X}$, so that $f$ is fully characterized on $\Delta_\f{X}$. The formula for $h_\Delta$ follows from the proof of Theorem \ref{thm:linearlearner}, where we showed that $h_\Delta=ev^+(\Delta_y)$ and corollary \ref{cor:psinverse-injective} since $\ev_\Delta$ is injective
\end{proof}



\subsection{Coordinates for Euclidean Learners} In the last section of this chapter, we will once and for all fix a Euclidean learner $\f{L}$ (as defined in \ref{def:euclideanlearner}) and introduce coordinates on the feature space $\f{X}$ and label space $\f{y}$. This will allow us to represent the learned hypothesis $h_\Delta$ by a matrix. We will show that this matrix indeed coincides with what is classically referred to as the \emph{regression matrix}.\\
Let us pick orthonormal bases $\r{E}\define (v_1,\ldots ,v_m)$ for $\f{X}$ and $\r{F}\define (w_1,\ldots w_n)$ for $\f{y}$. We will denote the associated coordinate maps by $\co_\r{E}$ and $\co_\r{F}$ respectively.\\
Now to any map $f \in \Hom_\d{R}(\f{X},\f{y})$, we'll associate the matrix $M_f\in \Mat_{n\times m}(\d{R})$ whose $i$-th column is given by $\co_\r{F}(v_i)$. It is well known that this matrix satisfies
\[ \co_{\r{F}}\big(f(v)\big) = M_f\cdot \co_{\r{E}}(v) \]
Our goal in this section is to compute the matrix $M_{h_\Delta}$ associated to the learned hypothesis $h_\Delta$ of the dataset $\Delta$. More precisely, we will prove the following theorem:
\begin{theorem}\label{thm:regressionwithcoordinates}
	 Let $\Delta=\big((x_1,y_1),\ldots ,(x_d,y_d)\big) \in \f{D}$ be a dataset.\\ Let $X = [\co_{\r{E}}(x_i)_i] \in \Mat_{m\times d}(\d{R})$ and $Y = [\co_{\r{F}}(y_j)_j]\in \Mat_{d \times n}(\d{R})$.\\ 
	 Then the matrix corresponding to the learned hypothesis $h_\Delta \in \Hom_{\d{R}}(\f{X},\f{y})$ is given by \[M_{h_\Delta} = Y\cdot X^+,\] where $X^+$ is the Moore-Penrose pseudo-inverse of $X$ (see \ref{lem:mppscoordinates}).\\ 
	 In other words, for any feature $v \in \f{X}$, we have\[
	\co_{\r{F}}\big(h_\Delta(v)\big) =\big( Y\cdot X^+ \big) \cdot \co_\r{E}(v)
	\]
\end{theorem}

We will prove this using a series of lemma's.\\ 
Since $h_\Delta = \ev_\Delta(\Delta_{\f{y}})$ by corollary \ref{cor:euclideanlearners}, it's worth reinterpreting the map $\ev_\Delta: \Hom(\f{X},\f{y})\mor \f{y}^d$ as a map between matrix spaces through the use of the orthonormal bases $\r{E}$ and $\r{F}$. To this end, we consider the isomorphism $M: \Hom(\f{X},\f{y})\mor \Mat_{n\times m}(\d{R})$which assigns to $f$ its matrix representation $M_f$,  after the choice of bases $\r{E}\subset \f{X}$ and $\r{F}\subset \f{y}$. We will also consider the map $\alpha_P$ the multiplication by a matrix $P$ on the right


\begin{lemma}\label{lem:lrcoorddiagram}
	Let $ X$ denote the matrix $\big[ \co_{\r{E}}(x_i)_i\big]$. Then the diagram:
	\begin{displaymath}
	\xymatrix{
	\Hom(\f{X},\f{y}) \ar[rr]^{\ev_\Delta}\ar[d]_{M_{(-)}}&& \f{y}^d\ar[d]^{(\co_{\r{F}})^d}\\
	\Mat_{n\times m}(\d{R})\ar[rr]_{\alpha_X}&&\Mat_{d\times n}(\d{R})	
	}
	\end{displaymath}
is commutative
\end{lemma}

\begin{proof}
	This is really just a restatement of the various definitions: \\Let $f\in \Hom(\f{X},\f{y})$. Then $\big(ev_\Delta \circ \co_\r{F}^d\big)(f)$ is the matrix $\big[(\co_{\r{F}}(f(x_i))_{i}\big]$, whose $i$-th column is the vector $\co_{\r{F}}\big(f(x_i)\big)\in \d{R}^n$.\\
	Going around the other way,  we obtain the matrix $\alpha_{X} \circ M_f=M_f\cdot X$, whose $i$-th column is $M_F\cdot \big[\co_{\r{E}}(x_i)\big]$. Now, by the definition of $M_f$, we have $M_f\cdot [\co_{\r{E}}(x_i)]=\co_{\r{F}}\big(f(x_i)\big)$, proving the claim

\end{proof}


Next, we will use the above lemma to describe the Moore-Penrose pseudo-inverse of $\ev_\Delta$ in terms of the maps $M$, $\alpha_X$ and $\big(\co_{\r{F}}\big)^d$. We'll need a little result from Euclidean geometry to do this..  recall that a map $f$ on an inner product space $V$ is \emph{orthogonal} if it preserves the inner product: 
\[
\bl{u}{v}=\bl{f(u)}{f(v)}
\]
This is equivalent to the adjoint coinciding with the inverse: $f^*=f^{-1}$. We now have the following linear algebraic lemma:
\begin{lemma}
Let $V$ and $W$ be finite dimensional inner product spaces.\\
	Let $f \in \Hom (V,W)$ and let $\phi \in \Hom_{\d{R}}(V,V)$ and $\psi \in \Hom_{\d{R}}(W,W)$ be orthogonal maps. Then the Moore-Penrose inverse of $\psi \circ f\circ \phi \in \Hom(V,W)$ is given by
	\[
	(\psi \circ f\circ \phi )^+ = \phi^{-1}\circ f^+ \circ \psi^{-1}
	\]
\end{lemma}

\begin{proof}
	We need to check the $4$ conditions of criterium $(3)$ in Lemma \ref{lem:mppschar}. First \[(\phi^{-1}\circ f^+ \circ \psi^{-1})\circ (\psi \circ f\circ \phi)\circ (\phi^{-1}\circ f^+ \circ \psi^{-1})=(\phi^{-1}\circ f^+\circ f\circ f^+\circ  \psi^{-1}) = \phi^{-1}\circ f^+ \circ \psi^{-1}\]
	The second condition is analogous.\\
	To prove the third condition, we invoke that $\phi$ and $\psi$ are orthogonal, so that ${\phi^{-1}}^*=\phi$ and ${\psi^{-1}}^*=\psi$. We now compute:
	\[
	\bigg((\phi^{-1}\circ f^+ \circ \psi^{-1})\circ (\psi \circ f\circ \phi) \bigg)^*=\bigg(\phi^{-1}\circ f^+ \circ f \circ \phi \bigg)^*=\phi^*\circ (f^+\circ f)^* \circ {\phi^{-1}}^*=\phi\circ (f^+\circ f)\circ \phi
	\]
	Where the last line follows from the orthogonality of $\phi$ and the fact that $f^+\circ f$ is self-adjoint since $f^+$ is the Moore-Penrose pseudo-inverse to $f$.\\
	The fourth condition is proven analogously.\\
\end{proof}

To use the above lemma we'll need to explain how $\co_{\r{F}}$ and $M$ are indeed orthogonal maps. For the benefit of the reader we first recap the inner products involved in the commutative diagram of Lemma \ref{lem:lrcoorddiagram}:
\begin{itemize}
	\item The bilinear form on $\Hom(\f{X},\f{y})$ does not have a general description, however  Lemma \ref{lem:innerproducts} implies that the maps 
	 \begin{equation}
	e_{i,j}(v_k)=
	\begin{cases}
	0, & \text{if}\ k \neq i  \\
	w_j, & \text{otherwise}
	\end{cases}
	\end{equation}
	form an orthonormal basis
	\item The space $\f{y}^d$ has the inner product defined by $\bl{(y_1,\ldots, y_n )}{(y'_1,\ldots y'_n)}=\sum_i \bl{y_i}{y'_i}$ and orthonormal basis $\big\{(w_{i_1},\ldots w_{i_m})\big\}_{i_1,\ldots i_m}$ by Lemma \ref{lem:innerproducts}
	\item The  matrix spaces are endowed with the \emph{Frobenius inner product}: for matrices $A$ and $B$, it's defined as $\bl{A}{B}=\sum_{i,j} A_{i,j}\cdot B_{i,j}$. It follows immediately that the elementary matrices $\big\{E_{i,j}\big\}_{i,j}$ form an orthonormal basis for these spaces.\\
	Another description of this inner product will be useful later on. Indeed, we have:
	\[
	\Tr(A\cdot B^t)=\sum_k(A\cdot B^t)_{k,k}=\sum_k(\sum_j A_{k,j}B^t_{j,k})=\sum_k(\sum_j A_{k,j}B_{k,j})=\bl{A}{B}
	\]
	This allows us to prove certain properties. For example, for any matrix $P$, we have
	\[
	\bl{A}{B\cdot P}=\Tr\big(A\cdot (B\cdot P)^t\big)=\Tr(A\cdot P^t\cdot B^t)=\bl{(A\cdot P^t)}{B}
	\]
\end{itemize}
We now have:

\begin{lemma}\label{lem:orthogonal}
	The maps $M$ and $(\co_{\r{F}})^d$ are orthogonal
\end{lemma}

\begin{proof}
	Indeed, it suffices to show that both maps $M$ and $\big(\co_{\r{F}}\big)^d$ send an orthonormal basis to an orthonormal basis. Now, the map $M$ sends the orthonormal basis $\big\{e_{i,j}\big\}_{i,j}$ to the orthonormal basis of elementary matrices $\big\{ E_{i,j}\big\}_{i,j}$.\\
	The map $(\co_{\r{F}})^d$ in turn sends the orthonormal basis $\big\{(w_{i_1},\ldots w_{i_m})\big\}_{i_1,\ldots i_m}$ to the elementary matrices $\big\{E_{i,j}\big\}_{i,j}$ as well.
\end{proof}

Combining both lemma's hence lets us write the Moore-Penrose inverse of $\ev_\Delta^+$ as
\[
\ev_\Delta^+ =\bigg( M^{-1} \circ \alpha_X\circ\co_{\r{F}}^d\bigg)^+= M^{-1}\circ \bigg( \alpha_X \bigg)^+\circ\co_{\r{F}}^d
\]
We now wish to simplify the map $\bigg(\alpha_X\bigg)^+$. To this end we note that for any matrix $P \in \Mat_{d\times m }(\d{R})$, we can associate its Moore-Penrose inverse as the inverse following example \ref{lem:mppscoordinates}
\begin{lemma}
	For any $P \in \Mat_{d\times m}(\d{R})$, let $\alpha_P$ denote the right multplication by $P$
	\[
	\alpha_P:\Mat_{n\times m}(\d{R})\mor \Mat_{d\times n}(\d{R}):A\fun A\cdot P
	\]
	 Then we have $\big( \alpha_P\big)^+ = \alpha_{P^+}$
\end{lemma}

\begin{proof}
We simply need to show that $\alpha_{P^+}$ satisfies the conditions of Moore-Penrose pseudo-inverse by checking $(3)$ of Lemma \ref{lem:mppschar}.\\
The first two conditions are absolutely trivial.\\ 
To prove the third condition, we need to show that the map $\alpha_{P}\circ \alpha_{P^+}$ (which coincides with $\alpha_{P^+\cdot P}$) is self-adjoint. To this end, we let $A,B \in \Mat_{n\times m}(\d{R})$. The discussion before Lemma \ref{lem:orthogonal} yields:
\[
\bl{A}{\alpha_{P^+\cdot P}(B)}=\Tr\bigg(A\cdot \big(B\cdot P^+\cdot P\big)^t\bigg)=\Tr\bigg(A \cdot (P^+\cdot P)^t\cdot B\bigg) = \Tr\bigg(A \cdot (P^+\cdot P)\cdot B^t\bigg)=\bl{\alpha_{P^+\cdot P}(A)}{B}
\]
Where the $3$rd equality follows from the fact that the matrix $P^+\cdot P$ is symmetric by Lemma \ref{lem:mppscoordinates}.\\
The proof of condition $(4)$ is completely analogous.
\end{proof}

The above lemma allows us to rewrite the Moore-Penrose pseudo-inverse one step further
\[
\ev_\Delta^+ = M_{}^{-1} \circ \alpha_{X^+} \circ \big( \co_{\r{F}}\big)^d
\]

We can now prove the main theorem of this section:

\begin{proof}[proof of Theorem \ref{thm:regressionwithcoordinates}]
	We want to show that $M_{h_\Delta} = Y\cdot X^+$. Now, we know from Corollary \ref{cor:euclideanlearners} that $h_\Delta = \ev_\Delta^+\big(\Delta_{\f{y}}\big) =\ev_\Delta^+\big(y_1,\ldots y_d\big)$. Moreover, the above discussion shows that 
\[
\ev_\Delta^+ = M^{-1} \circ \alpha_{X^+} \circ \big(\co_{\r{F}}\big)^d
\]
So that applying the set of features $(y_1,\ldots y_d)$ to the left hand side followed by the map $M$ yields
\[
M_{h_\Delta}=\bigg(\alpha_{X^+} \circ \co^d_\r{F}\bigg)(y_1,\ldots , y_d)=\alpha_{X^+}\bigg( \big[\co_{\r{F}}(y_1)\ldots \co_{\r{F}}(y_d)\big]\bigg)=\alpha_{X^+}(Y) = Y\cdot X^+
\]
as required
\end{proof}



\begin{definition}
We say that a learner is linear if it is of the above form
\end{definition}

One particular type of linear learner will be of particular interest, as it allows us to be a little more explicit with certain constructions: if we assume that $\f{X}$ itself carries the structure of a finite-dimensional inner product space, and put $\f{H}\define \Hom_{\d{R}}(\f{X},\f{y})$ as well as consider finite datasets $\Delta \subset \f{X}\times \f{y}$ such that $\Delta_{\f{X}}$ spans the whole of $\f{X}$
(where we recall our use of the notation \ref{not:Delta}), then it's easy to see that the conditions of Theorem \ref{thm:linearlearner} are satisfied so that we indeed obtain an example of a linear learner:


\begin{definition}\label{def:euclideanlearner}
A \emph{Euclidean learner} is a sharp (linear) learner where $\f{X},\f{y}$ are finite-dimensional inner product spaces, \[
\f{D} = \bigg\{ \Delta \subset \f{X}\times \f{y}\,\,\bigg\vert\,  \vert \Delta \vert \neq \infty , \textrm{ and span}(\Delta_{\f{X}}) = \f{X}\bigg\}
\text{,}\,\,\, \f{H}=\Hom_\d{R}(\f{X},\f{y})
\]
and 
\[
c:\f{D}\times \f{H}\mor \d{R}:(\Delta,f)\fun \vert \vert \big( y-f(x)\big)_{(x,y)\in \Delta} \vert \vert_{\f{y}^\Delta} 
\]
\end{definition}

\end{document}