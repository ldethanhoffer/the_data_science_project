%identitification
\NeedsTeXFormat{LaTeX2e}
\ProvidesPackage{preamble}[2020/04 custom preamble package for the data science project]

\begin{document}
\title{SUPPORT VECTOR MACHINES}

\maketitle	


\noindent\hrulefill
\tableofcontents
\noindent\hrulefill

\phantomsection
\label{section-phantom}

\section{Overview}
\section{Support Vector Machines}

\section{the Linearly Separable Case}


Let $V$ be a normed space and $f:V \mor \{-1,1\}$ a function. 
\begin{definition}
A support vector machine is a hyperplane $H \subset V$ which maximizes the tuple
\[
(d(H,f^{1}(-1),d(H,f^{-1}(1)))
\]	
\end{definition}

\begin{convention}
	We will always denote $f^{-1}(1)$ (resp. $f^{-1}(-1)$) by $V_1$ (resp. $V_{-1}$).
\end{convention}





\begin{theorem}
Assume that $V_{-1}, V_1	$ are linearly separable. Then there exists a unique support vector machine
\end{theorem}



\begin{lemma}
Let $H_1=a+V, H_2=b+V$ parallell hyperplanes. and $o \in V^{\perp}$
\[
d(H_1,H_2)=\frac{\vert \bl{o}{b-a}\vert}{\vert\vert o\vert \vert}
\] 	
\end{lemma}

\begin{proof}
Since $-b$ is an isometry, we can assume $b=0$. Now, for a vector $x\in a+V$, we have that $d(a+v,V)$	 is the distance to the orthogonal projection of $x$ on $V$. Since by definition, we have $x=\pi_V(x)+\pi_{V^\perp}(x)$, $d(x,V)=\vert \vert \pi_{V^{\perp}}(x)\vert\vert=\pi_{o}(x)$.\\
Now the projection of $x$ onto the line with direction $o$ is
\[
\frac{\vert \bl{o}{x}\vert}{\vert \vert o\vert \vert}=\frac{\vert \bl{o}{a}\vert}{\vert \vert o\vert \vert}\]
hence, the claim
\end{proof}


\begin{definition}
A linear separator is an affine function $V\mor \d{R}$ such that $f\cdot \sigma \ge 1$
\end{definition}

Since $\sigma$ is affine, the fibers $\sigma^{-1}(1)$ and $\sigma^{1}(-1)$ are parallel hyperplanes. This leads to the following definition

\begin{definition}
A support vector machine is a linear separator $\sigma$ which maximizes
\[
d(\sigma^{-1}(1),\sigma^{-1}(-1))
\]	
\end{definition}

\begin{theorem}
A support vector machine is equivalent to the data of a couple of vectors $(a,w)$such that $\bl{w}{x_i+a}\ge 1$ and $\vert \vert w\vert \vert$ is minimal	for this condition.
\end{theorem}

\begin{proof}
Since $\sigma$ is affine, the kernel is a hyperplane $a+V$. let $w in V^{\perp}$. Then	
\end{proof}

\end{document}