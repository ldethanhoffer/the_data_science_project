\documentclass{article}
\usepackage{amsmath,amsthm,amsfonts,amssymb,tikz,scalerel,mathrsfs,marvosym,titlesec,leftidx}

%for diagrams
\usepackage[all]{xy}

\usepackage{fancyhdr}
\usepackage{blkarray}
\usepackage{calrsfs}



%for referencing accross documents
\usepackage{xr-hyper}

\externaldocument[probability-]{probability}



%change the default font to bodoni
\usepackage[default]{gfsbodoni}
\usepackage[T1]{fontenc}

% Declare calligraphic font:
\DeclareMathAlphabet{\pr}{OMS}{zplm}{m}{n}


%start at section 0 for the overview: 
\setcounter{section}{-1}


%environments
%
\theoremstyle{plain}
\newtheorem{corollary}{Corollary}[section]
\newtheorem{convention}[corollary]{Convention}
\newtheorem{example}[corollary]{Example}
\newtheorem*{theorem*}{theorem}
\newtheorem{theorem}[corollary]{Theorem}
\newtheorem{lemma}[corollary]{Lemma}
\newtheorem*{lemma*}{Lemma}
\newtheorem{maintheorem}{Theorem}
%reset maintheorem counter to use the corresponding letter
\renewcommand{\themaintheorem}{\Alph{maintheorem}}
\newtheorem{remark}[corollary]{Remark}
\newtheorem{proposition}[corollary]{Proposition}


\theoremstyle{definition}
\newtheorem{definition}[corollary]{Definition}
\newtheorem{definition*}{Definition}
\newtheorem{notation}{Notation}

%macros
%
\newcommand{\A}{\mathcal{A}}
\DeclareMathOperator{\avg}{avg}
\newcommand{\abs}[1]{\vert #1 \vert }
\DeclareMathOperator{\ad}{ad}
\DeclareMathOperator{\Aff}{aff}
\DeclareMathOperator{\arginf}{arginf}
\DeclareMathOperator{\argmin}{argmin}
\DeclareMathOperator{\argsup}{argsup}
\DeclareMathOperator{\argmax}{argmax}
\DeclareMathOperator{\Aut}{Aut}
\renewcommand{\b}{\bullet}
\newcommand{\bl}[2]{\left\langle #1\,,#2\, \right\rangle}
\newcommand{\bp}{\mathbf{\Pi}}
\def\BVm{\operatorname{BV}_-}
\DeclareMathOperator{\BiMod}{BiMod}
\DeclareMathOperator{\bimod}{bimod}
\DeclareMathOperator{\card}{card}
\newcommand{\C}{{\mathcal{C}}}
\DeclareMathOperator{\CC}{CC}
\def\cd{\operatorname {cd}}
\DeclareMathOperator{\ch}{ch}
\DeclareMathOperator{\CH}{CH}
\DeclareMathOperator{\Conv}{Conv}
\DeclareMathOperator{\co}{co}
\DeclareMathOperator{\coh}{coh}
\DeclareMathOperator{\cone}{cone}
\DeclareMathOperator{\coker}{coker}
\renewcommand{\d}[1]{\mathbb{#1}}
\newcommand{\D}{{\mathscr{D}}}
\DeclareMathOperator{\der}{R}
\DeclareMathOperator{\Def}{Def}
\DeclareMathOperator{\Dir}{Dir}
\newcommand{\define}{\stackrel{\operatorname{def}}{=}}
\let\div\relax %unassign \div
\DeclareMathOperator{\div}{div}
\DeclareMathOperator{\divu}{div}
\newcommand{\draw}[1]{\widetilde{\textrm{S}} #1}
\newcommand{\ds}{\oplus}
\newcommand{\DS}{\bigoplus}
\newcommand{\epi}{\xymatrix{{}\ar@{->>}[r]&{}}}
\DeclareMathOperator{\End}{End}
\DeclareMathOperator{\Entropy}{Entropy}
\DeclareMathOperator{\Ext}{Ext}
\newcommand{\f}[1]{\mathfrak{#1}}
\DeclareMathOperator{\fd}{fd}
\newcommand{\fC}{\f{C}}
\newcommand{\fD}{\f{D}}
\newcommand{\floor}[1]{\left\lfloor #1 \right\rfloor}
\newcommand{\fun}{\mapsto}
\newcommand{\norm}[1]{\vert \vert #1 \vert \vert}
\DeclareMathOperator{\FRel}{FRel}
\newcommand{\dgg}{{\f{g}^\bullet}}
\newcommand{\g}{\f{g}}
\DeclareMathOperator{\Gd}{Gd}
\DeclareMathOperator{\ev}{ev}
\DeclareMathOperator{\gldim}{gl.dim}
\let\H\relax %unassign \H
\DeclareMathOperator{\H}{H}
\DeclareMathOperator{\HH}{HH}
\DeclareMathOperator{\HC}{HC}
\DeclareMathOperator{\Hom}{Hom}
\DeclareMathOperator{\Id}{Id}
\DeclareMathOperator{\ID3}{ID_3}
\DeclareMathOperator{\im}{im}
\newcommand{\iso}{\stackrel{\sim}{\longrightarrow}}
\DeclareMathOperator{\Jac}{Jac}
\renewcommand{\k}{\Bbbk}
\DeclareMathOperator{\kNN}{kNN}
\newcommand{\K}{\mathbb{K}}
\DeclareMathOperator{\length}{length}
\DeclareMathOperator{\lrk}{\l.rk}
\newcommand{\leftperp}[1]{\leftidx{^\perp}{#1}}
\DeclareMathOperator{\LS}{LS}
\newcommand{\m}{\f{m}}
\DeclareMathOperator{\Mat}{Mat}
\DeclareMathOperator{\MCM}{MCM}
\DeclareMathOperator{\MC}{MC}
\DeclareMathOperator{\MCcal}{\mathcal{MC}}
\DeclareMathOperator{\MLE}{MLE}
\DeclareMathOperator{\Mod}{Mod}
\DeclareMathOperator{\Mor}{Mor}
\def\mod{\operatorname{mod}}
\newcommand{\mor}{\longrightarrow}
\newcommand{\T}{\mathscr{T}}
\renewcommand{\O}{\mathcal{O}}
\DeclareMathOperator{\NS}{NS}
\DeclareMathOperator{\Num}{Num}
\newcommand{\p}{\mathfrak{p}}
\DeclareMathOperator{\per}{per}
\DeclareMathOperator{\Ob}{Ob}
\DeclareMathOperator{\Perf}{Perf}
\DeclareMathOperator{\Pic}{Pic}
\DeclareMathOperator{\Proj}{Proj}
\DeclareMathOperator{\poly}{poly}
\DeclareMathOperator{\pred}{pred}
\DeclareMathOperator{\Qcoh}{Qcoh}
\DeclareMathOperator{\QGr}{QGr}
\DeclareMathOperator{\Rel}{Rel}
\DeclareMathOperator{\RHom}{RHom}
\renewcommand{\r}[1]{\mathcal{#1}}
\DeclareMathOperator{\rk}{rk}
\DeclareMathOperator{\rrk}{r.rk}
\DeclareMathOperator{\rsample}{RS}
\def\locExt{\mathscr{E}\mathit{xt}}
\newcommand{\sample}[1]{\textrm{S} #1}
\DeclareMathOperator{\SL}{SL}
\DeclareMathOperator{\RS}{RS}
\DeclareMathOperator{\SPS}{SPS}
\DeclareMathOperator{\sing}{sing}
\DeclareMathOperator{\sep}{sep}
\DeclareMathOperator{\Test}{Test}
\DeclareMathOperator{\test}{test}
\DeclareMathOperator{\Train}{Train}
\DeclareMathOperator{\train}{train}
\DeclareMathOperator{\true}{true}
\DeclareMathOperator{\vectorspan}{span}
\DeclareMathOperator{\Spec}{Spec}
\DeclareMathOperator{\Supp}{Supp}
\DeclareMathOperator{\SVM}{SVM}
\DeclareMathOperator{\SupVec}{SupVec}
\DeclareMathOperator{\Sym}{Sym}
\def\locHom{\operatorname {\mathscr{H}\mathit{om}}}
\DeclareMathOperator{\Tails}{Tails}
\def\ltr{\overset{L}{\tr}}
\renewcommand{\c}[1]{\mathcal{#1}}
\newcommand{\mono}{\xymatrix{{}\ar@{^{(}->}[r]&{}}}\DeclareMathOperator{\num}{num}
\DeclareMathOperator{\rad}{rad}
\DeclareMathOperator{\Tr}{Tr}
\DeclareMathOperator{\Tor}{Tor}
\DeclareMathOperator{\Tors}{Tors}
\newcommand{\tr}{\otimes}
\newcommand{\trps}[1]{ \leftidx{^{\operatorname{tr}}}{#1}}
\def\uRHom{\operatorname {R\mathcal{H}\mathit{om}}}
\DeclareMathOperator{\Var}{Var}
\newcommand{\w}{{\tt{w}}}
\newcommand{\Q}{\mathbb{Q}}
\DeclareMathOperator{\qis}{qis}
\DeclareMathOperator{\Qvr}{\mathcal{Q}}
\newcommand{\Z}{\mathbb{Z}}


%add dots between section and page in toc
\usepackage{tocloft}
\renewcommand{\cftsecleader}{\cftdotfill{\cftdotsep}}


%imake nterline a little larger
\linespread{1.28}

%change the dimensions of the margin
\usepackage{geometry}
 \geometry{
 a4paper,
 total={210mm,297mm},
 left=23mm,
 right=23mm,
 top=20mm,
 bottom=20mm,
 }
 
%changeqedsymbol
\renewcommand{\qedsymbol}{\CrossedBox}

%make the (sub)subsections run into the text, the 0.25 is the distance between the numbering and title
\titleformat{\subsection}[runin]{\normalfont\Large\bfseries}{\thesubsection.}{0.25em}{}
\titleformat{\subsubsection}[runin]{\normalfont\normalsize\bfseries}{\thesubsubsection.}{0.25em}{}

% remove indentation:
\setlength\parindent{0pt}

\begin{document}
\title{\textbf{MATHVSMACHINE: THE BOOK}}
\author{LOUIS DE THANHOFFER DE VOLCSEY}
\date{}
\maketitle	


\noindent\hrulefill
\tableofcontents{}
\noindent\hrulefill

\section{the Normal equation}


\noindent In this section, we'll refresh the reader on projections onto images of linear maps: The starting point of this construction is the following lemma:
\begin{lemma}\label{lem:mindist-perp}
	Let $W\subset V$ be a subspace of the inner product space $V$. Let $v \in V$ and $u \in W$.\\
	Then the  following are equivalent:
	\begin{enumerate}
		\item $(v-u) \perp W$
		\item $	\argmin_{w \in W}\norm{v-w} =u$	
	\end{enumerate}
\end{lemma}

\begin{proof}
	Let $w \in W$.\\
	Assuming $u \in W$ satisfies the first condition, we have $v-u \perp u-w$ and by the Pythagorean theorem, we have:
	\[
	\norm{v-w}^2 = \norm{v-u}^2+\norm{u-w}^2 \ge \norm{v-u}^2
	\]
	Proving the second condition.\\ 
	Conversely, we assume $u\in W$ satisfies the second condition and apply the following trick:\\
	Consider the function:
	\[
	\phi:\d{R}\mor \d{R}: t \fun \norm{v-u+tw}^2
	\]	
	Since $\norm{v-u}^2$ is minimal, $\phi$ has a minimum at $t=0$. Moreover, $\phi$ is differentiable so that $\phi'(0)=0$. It's also easy to see that
	\[
	\phi(t) = \norm{v-u}^2 +2\cdot t \bl{v-u}{w}+t^2\norm{w}^2
	\]
	Hence $0=\phi'(0) = 2 \bl{v-u}{w}$, and $v-u\perp w$ as required
\end{proof}

\begin{lemma}\label{lem:projection}
	Let $W \subset V$ be a  subspace of a finite-dimensional inner product space. Then the map
	\[
	\pi: V\mor W: v \fun \argmin_{w \in W} \norm{v- w}
	\]
	is well defined
\end{lemma}

\begin{proof}
	By the above lemma we need to show that for any $v \in V$, there exists a unique $\pi(v) \in W$ such that $v-\pi(v) \perp W$. To this end, we look at the following map $ f \in W^*$
	\[
	f: W \mor \d{R}: w \fun \bl{v}{w}
	\]
	Since $W^*$ is in turn an inner product space, $f$ can be written in the form $\bl{\pi(v)}{-}$ for some unique $\pi(v) \in W$. The result now follows
\end{proof}

The lemma above motivates the following definition:

\begin{definition}\label{def:projection}
	Let $W\subset V$ be a subspace of a finite dimensional inner product space. Then the unique map $\pi: V\mor W$ defined by 
	\[
	\pi(v) \define \argmin_{w \in W}\norm{v-w}
	\]
	is \emph{the projection of $V$ onto $W$}
\end{definition}
\begin{corollary}\label{lem:projcoincide}
	Let $W=\im(f)\ds \im(f)^{\perp}$ and $\pi_{\im(f)}:W\mor \im(f)$ be the canonical projection. Then $\pi_{\im(f)}$ coincides with the projection onto the subspace $\im(f)\subset W$ in the sense of Definition \ref{def:projection}
\end{corollary}

\begin{proof}
This follows immediately from \ref{lem:mindist-perp}
\end{proof}
\noindent Next, we consider a linear map $f \in \Hom_\k(U,V)$ and let $W$ be the subspace $\im(f)$. One can give a more explicit description of the projection $\pi: V\mor W$:

\begin{lemma}\label{lem:normaleq}
	Let $f \in \Hom(U,V)$ and and $v \in V$ Then the following are equivalent:
	\begin{enumerate}
		\item	$f(u)$ is the projection of $v$ onto the subspace $\im(f)\subset V$
		\item The vector $u \in U$ satisfies $(f^* \circ f)(u) = f^*(v)$	
	\end{enumerate}
\end{lemma}

\begin{proof}
	$f(u)$ is the projection of $v$ onto $\im(f)$ if and only if $\bl{v-f(u)}{f(u')}$ for any $u' \in U$ by Lemma \ref{lem:mindist-perp}. Now,
	\[
	\bl{v-f(u)}{f(u')}=\bl{f^*(v)-f^*f(u)}{u'}
	\]
	This last expression is $0$ if and only if $f^*(v)-f^*f(u)=0$ since the inner product is nondegenerate
\end{proof}

The above lemma justifies the following definition:

\begin{definition}
	Let $f \in \Hom(V,W)$ and $w \in W$.\\ 
	We say that $v \in V$ satisfies the normal equation if and only if 
	\[
	(f^* \circ f)(v) = f^* w
	\]
\end{definition}
In this terminology, we can restate lemma \ref{lem:normaleq} as follows:
\begin{lemma}
	Let $f \in \Hom(V,W)$ and $w \in W$ Then the following are equivalent:
	\begin{enumerate}
		\item $w =\argmin_{u \in V}\vert \vert w-f(u)\vert \vert$
		\item $v$ is a solution to the normal equation $(f^* \circ f)(v) = f^* w$
	\end{enumerate}
\end{lemma}
\noindent It is finding the solutions to this equation that we are interested in. It turns out that one can give an explicit description of them using the so-called \emph{Moore-Penrose pseudo-inverse}. Since this construction seems to be a little less well covered in standard linear algebra literature, we'll discuss in detail below: 
\section{the (Moore-Penrose) Pseudo-inverse}

In this section, we will let $V, W$ be finite-dimensional vector spaces and $f \in \Hom_\d{R}(V,W)$.\\ It is well-known that $f$ does not have an inverse in general. There is however a natural generalization of the notion of inverse which can be defined for \emph{any} map: a \emph{pseudo-inverse}. More precisely, if $f$ either has  a nonzero kernel or if the image of $f$ is not the whole of $W$, then the inverse of $f$ will not exist. One natural way to remediate this issue is to consider complements for both subspaces and write 
\[V\define \ker(f)\ds U_V \textrm{ and } W\define \im(f)\ds U_W\]

It's easy to see that restricting $f$ to appropriate subspaces now does produce an invertible map as follows:

\begin{lemma}
	the map $f: U_V\mor \im(f)$ is an isomorphism.
\end{lemma}
\noindent We'll denote the inverse of $f$ on $U_V$ by $f^\sharp:\im(f)\mor U_V$. A pseudo-inverse is now the natural lift of $f^\sharp$ to the whole of $W$:


\begin{lemma}\label{lem:pseudo-inverse}
	There exists a unique map $f^\sharp:W\mor U_V$ making the following diagram commute:
	\begin{displaymath}
	\xymatrix{
	W\ar[d]_{\pi_{\im(f)}}\ar[drr]^{f^\sharp}\\
	\im(f)\ar[rr]_{f^\sharp} && U_V
	}
	\end{displaymath}
\end{lemma}

\begin{proof}
The commutativity of the diagram means that for $u \in U_V$, we have
\[
	f^\sharp(w) \define u\iff f^\sharp(\pi_{\im(f)}(w)) =u\iff \pi_{\im(f)}(w) =f(u)
\]
Where the second equivalence follows from the fact that $f^\sharp$ is the inver of $f$ on $U_V$.\\
The claim will thus follow if we show that the above assignment is indeed a well-defined linear map. To this end assume that $u,u' \in U_V$ satisfy $f(u')=\pi_{\im(f)}(w)=f(u)$.\\ Then $u-u' \in \ker(f)$, hence $u-u' \in \ker(f)\cap U_V$ in particular. Now since $\ker(f)\ds U_V =V$, we have  $u-u'=0$, so that $u=u'$, showing the well-definedness.\\ 
We leave the linearity to the reader.
\end{proof}
It will be helpful to note that the map $f^\sharp\in \Hom(W,V)$ can also be characterized by $\im(f^\sharp)\subset U_V$ and $f\circ f^\sharp =\pi_{\im(f)}$.\\
To give the map $f^\sharp$ a name, we first let $\Lambda(f)$ denote the set
\[
\Lambda(f)\define \{(U_V,U_W)\vert \, \ker(f)\ds U_V=V \textrm{ and } \im(f)\ds U_W =W \}
\]
and conclude from Lemma \ref{lem:pseudo-inverse} that there is a assignment:
\[
\Phi: \Lambda(f)\mor \Hom_\d{R}(W,V):(U_V,U_W)\fun f^\sharp
\]
where $f^\sharp \in \Hom_{\d{R}}(W,V)$ is the unique map satisfying
\[
f \circ f^\sharp = \pi_{\im(f)} \textrm{ and } \im(f^\sharp)\subset U_V
\]
Let's denote the image of $\Phi$ by $\Pi(f)$. Summarizing the discussion, we make the following:
\begin{definition}\label{def:ps}
Let $(U_V,U_W) \in \Lambda(f)$. Then the pseudo-inverse of $(U_V,U_W,f)$ is the map $\Phi(f)$.\\
We say that $g\in \Hom_{\d{R}}(W,V)$ is a pseudo-inverse to $f$ if $g\in \Pi(f)$
\end{definition}

We can give a slightly different description of pseudo-inverses by describing them on the 2 components in the decomposition $\im(f)\ds U_W =W$:


\begin{lemma}\label{lem:pschar2}
Let $(U_V,U_W)$ in $\Lambda(f)$. Then the following are equivalent:
\begin{enumerate}
	\item $f^\sharp$ is the pseudo-inverse to $(U_V,U_W,f)$
	\item $f^\sharp\arrowvert_{\im(f)}$ is the inverse to $f:U_V\mor \im(f)$ and $f^\sharp\arrowvert_{U_W}=0$
\end{enumerate}

\end{lemma}

\begin{proof}
	Since the pseudo=inverse to $(U_V,U_W,f)$ is unique, it suffices to show that the pseudo-inverse indeed satisfies the conditions of $(2)$. The fact that $f^\sharp\arrowvert_{\im(f)}$ is the inverse of $f\arrowvert_{U_V}$ follows from 
	\[
	(f\circ f^\sharp) \arrowvert_{\im(f)}=\big(\pi_{\im(f)}\big)\arrowvert_{\im(f)}=\Id\arrowvert_{\im(f)}
\] 
Moreover, if $w \in U_W$, then $\pi_{\im(f)}(w)=0$ since $\im(f)\ds U_W$. Hence $f^\sharp(w)=f^\sharp(\pi_{\im(f)}(w))=0$ by Lemma \ref{lem:pseudo-inverse}
\end{proof}
Our next order of business is to give an explicit description of the set $\Pi(f)$ of pseudo-inverses to $f$. We begin by showing that we can describe the complements $U_V$ and $U_W$ solely by using the maps $f$ and $f^\sharp$:
\begin{lemma}\label{lem:psim-ker}
	Let $f^\sharp$ be the pseudo-inverse to $(U_V,U_W,f)$. Then $U_V=\im(f^\sharp)$ and $U_W=\ker(f^\sharp)$
\end{lemma}

\begin{proof}
	We have $\im(f^\sharp)\subset U_V$ by Definition \ref{def:ps} . Moreover, $f^\sharp$ is a composition of surjections and hence itself surjective, proving the first claim.\\
	To prove the second claim, note that the second condition of Lemma \ref{lem:pschar2} immediately implies that $U_W\subset \ker(f^\sharp)$. We can also show the other inclusion by assuming that $w\in W$ satisfies $f^\sharp(w)=0$, in which case $\pi_{\im(f)}(w)=f(f^\sharp(w))=f(0)=0$, implying that $w$ lies in the component $U_W$ of the decomposition $\im(f)\ds U_W=W$ as required
\end{proof}

Taking the above lemma one step further allows us to describe the set $\Pi(f)$ of pseudo-inverses as promised:

\begin{lemma}\label{lem:charpi}
	Let $f \in \Hom(V,W)$. Then the following are equivalent:
	\begin{enumerate}
	\item $g \in \Pi(f)$
	\item $(f\circ g)\arrowvert_{\im(f)}=\Id$ and $(g\circ f)\arrowvert_{\im(g)}=\Id$ 
	\end{enumerate}
\end{lemma}

\begin{proof}
	Let $g$ be a pseudo-inverse to $f$ and define $U_V\define\im(g)$ and $U_W\define\ker(f)$. Then Lemma \ref{lem:psim-ker} shows that $g$ is in fact the pseudo-inverse to the triple $\big(U_V,U_W,f\big)$. Now, since $g\arrowvert_{\im_(f)}$ is the inverse to $f\arrowvert_{U_V}$ by Lemma \ref{lem:pschar2}, we have $(f \circ g)\arrowvert_{\im(f)}=\Id$ and $(g \circ f)\arrowvert_{\im(g)}=(g \circ f)\arrowvert_{U_V}=\Id$.\\
	Conversely, assume that $g$ satisfies the conditions in (2).\\
	We begin by showing that $\big(\im(g),\ker(g)\big) \in \Lambda(f)$. Let's show  that $\im(f)\ds \ker(g)=W$ by way of example. Indeed, first note that $\im(f)\cap \ker(g)=0$, as any $w$ in this intersection must satisfy $w=(f\circ g)(w)=f(0)=0$. Moreover, if we write $w=\big(w-f( g(w))\big)+f ( g(w)\big)$, we see that trivially $f(g(w))\in \im (f)$ and \[
	g(w-f(g(w)))=g(w)-(g(f(g(w))=g(w)-g(w)=0
	\]
	so that $\big(w-f(g(w))\big) \in \ker(g)$. This indeed shows that $\im(f)\ds \ker(g)=W$. The proof of $\im(g)\ds \ker(f)=V$ is completely analogous, allowing us to conclude that $(\im(g),\ker(g))\in \Lambda(f)$.\\
	It now remains to show that $g$ is indeed a pseudo-inverse to the triple $(\im(g),\ker(f),f)$. By Lemma \ref{lem:pschar2}, it suffices to show that $g\arrowvert_{\im(f)}$ is the inverse to $f\arrowvert_{\im(g)}$ and that $g\arrowvert_{\ker(g)}=0$. The first claim follows immediately from the fact that $g$ is a left inverse to $f:\im(g)\mor W$ and the second claim is trivial.
\end{proof}

\noindent In order to summarize the previous 2 lemmas, we introduce the following assignment, which is well-defined by Lemma \ref{lem:psim-ker}

\[
\Psi: \Pi(f)\mor \Lambda(f): g\fun \big(\im(g),\ker(g)\big)
\]
We now have:
\begin{lemma}\label{lem:psinverses}
	Let $f \in \Hom(V,W)$. Then:
	\begin{itemize}
		\item 
		$\Pi(f)=\big\{ g \in \Hom(W,V)\,\,\big\vert\,\, (f\circ g)\arrowvert_{\im(f)}=\Id$ \textrm{ and } $(g\circ f)\arrowvert_{\im(g)}=\Id \big\} $
		\item The assignments $\Phi$ and $\Psi$ define 1:1 correspondences between $\Lambda(f)$ and $\Pi(f)$
	\end{itemize}
\end{lemma}


\begin{proof}
	The first claim simply restates Lemma \ref{lem:charpi}. To prove the second, we note that $\Psi\circ \Phi=\Id$  by Lemma \ref{lem:psim-ker}. Moreover, $\Phi$ is surjective by definition, implying that $\Phi\circ \Psi =\Id$ as well
\end{proof}



\noindent We finish our discussion of pseudo-inverses by discussing a special choice of pseudo-inverse in $\Pi(f)$ that one can make if the vector spaces $V$ and $W$ are equipped with inner products. Indeed, recall the following standard result:

\begin{lemma}
 Let $U \subset V$ be a subspace of a finite dimensional inner product space. Then $U\ds U^{\perp}=V$
\end{lemma}

This leads us to the following Definition:
\begin{definition}\label{def:mpinverse}
Let $V,W$ be finite-dimensional inner product spaces and let $f \in \Hom_\d{R}(V,W)$. Then the \emph{Moore-Penrose pseudo-inverse} is the pseudo-inverse to the triple $(\ker(f)^\perp,\im(f)^\perp,f)$.\\ We will denote it by $f^+$
\end{definition}

\noindent It turns out that we can give a very satisfying description of Moore-Penrose pseudo-inverses:

\begin{lemma}\label{lem:mppschar}
	Let $V, W$ be finite-dimensional inner product spaces and $f \in \Hom(V,W)$. Then the following are equivalent:
	\begin{enumerate}
		\item $g$ is the Moore-Penrose pseudo-inverse $f^+$ to $f$
		\item $g$ is a pseudo-inverse to $f$ and $g\circ f$ and $f \circ g$ are self-adjoint linear maps
		\item $f$ and $g$ satisfy $f\circ g \circ f=f$, $g\circ f\circ g =g$,  $(g\circ f)^*=g\circ f$ and $(f\circ g)^* =f\circ g$
	\end{enumerate}
\end{lemma}

\begin{proof}
	The equivalence $(2)\iff (3)$ is simply a restatement of Lemma \ref{lem:psinverses}.\\
	We now prove $(2)\implies (1)$:\\
	Assume that $g$ is a pseudo-inverse to $f$ and that $g\circ f$ and $f \circ g$ are both self-adjoint. then Lemma \ref{lem:psim-ker} implies that $g$ is the pseudo-inverse to the triple $\big(\im(g),\ker(f),f\big)$. The claim will thus follow if we show that $\im(g)=\ker(f)^\perp$ and $\ker(g)=\im(f)^\perp$. By way of example, we will prove the former equality: First note that since $\im(g)\ds \ker(f)=V$, it suffices to show that $\im(g)\perp \ker(f)$. Indeed, for $w \in W$ and $v \in \ker(f)$, we have: 
	\[
	\bl{v}{g(w)}=\bl{v}{(g\circ f)(g(w))}=\bl{(g\circ f)^*(v)}{g(w)}=\bl{(g\circ f)(v)}{g(w)}=\bl{g(0)}{g(w)}=0
	\]
	The proof of $\ker(g)=\im(f)^\perp$ is analogous.\\
	Finally, we show $(1)\implies (2)$:\\
	Assume that $g$ is the Moore Penrose pseudo-inverse to $f$. Ie $g$ is the  pseudo-inverse to the triple $(\ker(f)^\perp,\im(f)^\perp,f)$. We will show that $(f\circ g)$ is self-adjoint and leave the other claim to the reader. To this end, let $v,v' \in V$. Then
	\begin{align*}
	\bl{v}{g(f(v'))}&=\bigg\langle v-g(f(v))+g(f(v)), g(f(v'))-v'+v'\bigg\rangle\\
	&= \bigg\langle v-g(f(v)), g(f(v'))\bigg\rangle+\bigg\langle g(f(v)),g(f(v'))-v'\bigg\rangle+\bigg\langle g(f(v)), v'\bigg\rangle
	\end{align*}
	Now, since $f\circ g\circ f=f$, we conclude that $v-g(f(v))$ and $g(f(v'))-v'$ lie in $\ker(f)$. Moreover, since $\ker(f)=\im(g)^\perp$, we conclude that 
	\[
	\bl{v-g(f(v))}{g(f(v'))}=\bl{g(f(v))}{g(f(v'))-v'}=0
	\]
	So that 
	\[
	\bl{v}{g(f(v'))}=\bl{g(f(v))}{v'}
	\] implying that $f\circ g=(f\circ g)^*$. The equality $g\circ f = (g\circ f)^*$ is completely analogous.
\end{proof}

As mentioned in the introduction of this section, our main motivation for studying the Moore-Penrose pseudo-inverse, is to provide a description of the projection of a vector onto the image of a linear map. We begin with the following preparatory lemma:



\begin{lemma}\label{lem:proj=mppsinverse}
	Let $V,W$ be finite-dimensional inner product spaces and  $f \in \Hom(V,W)$. Let $v \in V$ and $w\in W$. Finally denote the Moore-Penrose inverse of $f$ by $f^+$. Then the following are equivalent:
	\begin{enumerate}
		\item $f(v)$ is the projection of $w$ onto the subspace $\im(f)$
		\item $v$ satisfies the normal equation $(f^*\circ f)(v)=f^*(w)$
		\item $v$ lies in the affine subspace $f^+(w) + \ker(f)$
	\end{enumerate}
\end{lemma}

\begin{proof}
	The equivalence of $(1)\iff (2)$ is simply a restatement of Lemma \ref{lem:normaleq}.\\
	To show the equivalence of $(1)\iff (3)$, we first note that $f(f^+w))=\pi_{\im(f)}$, where $\pi_{im(f)}$ is the projection onto the subspace $\im(f)\subset W$ by Lemma \ref{lem:projcoincide}. This shows that the vector $f^+(w) \in V$ indeed satisfies the condition $(1)$. Next, assume (1), so that $v \in V$ satisfies $f(v)=\pi_{\im(f)}(v)$ and write $v= f^+(w)+v'$. Then\[
	f(v)=\pi_{\im(f)(w)}\iff f(f^+(w)+v')=\pi_{\im(f)}(w)\iff \pi_{\im(f)}(w)+f(v')=\pi_{\im(f)}(w)\iff v'\in \ker(f)
	\]
	This proves the claim
\end{proof}

This lemma has an interesting corollary which allows us to write the Moore-Penrose even more explicitly which will play an important role later on:

\begin{corollary}\label{cor:psinverse-injective}
Let $V$ be a finite dimensional vector space and $W$ a finite dimensional inner product space. Let $f \in \Hom(V,W)$ be injective and choose \emph{any} inner product on $V$. Then
\[
f^+ = (f^*\circ f)^{-1}\circ f^*
\] 
\end{corollary}

\begin{proof}
	Since $f$ is injective (so that $\ker(f)=0$), $f^+$ is the pseudo-inverse to the triple $(V,\im(f)^{\perp},f)$ by Definition \ref{def:mpinverse}. It follows immediately that this condition is independent of the inner product on $V$. To prove the formula, simply note that $f^*\circ f$ is invertible if $f$ is injective and apply the second criterium of Lemma \ref{lem:proj=mppsinverse}
\end{proof}