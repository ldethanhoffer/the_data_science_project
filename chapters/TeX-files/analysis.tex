\documentclass{article}
\usepackage{amsmath,amsthm,amsfonts,amssymb,tikz,scalerel,mathrsfs,marvosym,titlesec,leftidx}

%for diagrams
\usepackage[all]{xy}

\usepackage{fancyhdr}
\usepackage{blkarray}
\usepackage{calrsfs}



%for referencing accross documents
\usepackage{xr-hyper}

\externaldocument[probability-]{probability}



%change the default font to bodoni
\usepackage[default]{gfsbodoni}
\usepackage[T1]{fontenc}

% Declare calligraphic font:
\DeclareMathAlphabet{\pr}{OMS}{zplm}{m}{n}


%start at section 0 for the overview: 
\setcounter{section}{-1}


%environments
%
\theoremstyle{plain}
\newtheorem{corollary}{Corollary}[section]
\newtheorem{convention}[corollary]{Convention}
\newtheorem{example}[corollary]{Example}
\newtheorem*{theorem*}{theorem}
\newtheorem{theorem}[corollary]{Theorem}
\newtheorem{lemma}[corollary]{Lemma}
\newtheorem*{lemma*}{Lemma}
\newtheorem{maintheorem}{Theorem}
%reset maintheorem counter to use the corresponding letter
\renewcommand{\themaintheorem}{\Alph{maintheorem}}
\newtheorem{remark}[corollary]{Remark}
\newtheorem{proposition}[corollary]{Proposition}


\theoremstyle{definition}
\newtheorem{definition}[corollary]{Definition}
\newtheorem{definition*}{Definition}
\newtheorem{notation}{Notation}

%macros
%
\newcommand{\A}{\mathcal{A}}
\DeclareMathOperator{\avg}{avg}
\newcommand{\abs}[1]{\vert #1 \vert }
\DeclareMathOperator{\ad}{ad}
\DeclareMathOperator{\Aff}{aff}
\DeclareMathOperator{\arginf}{arginf}
\DeclareMathOperator{\argmin}{argmin}
\DeclareMathOperator{\argsup}{argsup}
\DeclareMathOperator{\argmax}{argmax}
\DeclareMathOperator{\Aut}{Aut}
\renewcommand{\b}{\bullet}
\newcommand{\bl}[2]{\left\langle #1\,,#2\, \right\rangle}
\newcommand{\bp}{\mathbf{\Pi}}
\def\BVm{\operatorname{BV}_-}
\DeclareMathOperator{\BiMod}{BiMod}
\DeclareMathOperator{\bimod}{bimod}
\DeclareMathOperator{\card}{card}
\newcommand{\C}{{\mathcal{C}}}
\DeclareMathOperator{\CC}{CC}
\def\cd{\operatorname {cd}}
\DeclareMathOperator{\ch}{ch}
\DeclareMathOperator{\CH}{CH}
\DeclareMathOperator{\Conv}{Conv}
\DeclareMathOperator{\co}{co}
\DeclareMathOperator{\coh}{coh}
\DeclareMathOperator{\cone}{cone}
\DeclareMathOperator{\coker}{coker}
\renewcommand{\d}[1]{\mathbb{#1}}
\newcommand{\D}{{\mathscr{D}}}
\DeclareMathOperator{\der}{R}
\DeclareMathOperator{\Def}{Def}
\DeclareMathOperator{\Dir}{Dir}
\newcommand{\define}{\stackrel{\operatorname{def}}{=}}
\let\div\relax %unassign \div
\DeclareMathOperator{\div}{div}
\DeclareMathOperator{\divu}{div}
\newcommand{\draw}[1]{\widetilde{\textrm{S}} #1}
\newcommand{\ds}{\oplus}
\newcommand{\DS}{\bigoplus}
\newcommand{\epi}{\xymatrix{{}\ar@{->>}[r]&{}}}
\DeclareMathOperator{\End}{End}
\DeclareMathOperator{\Entropy}{Entropy}
\DeclareMathOperator{\Ext}{Ext}
\newcommand{\f}[1]{\mathfrak{#1}}
\DeclareMathOperator{\fd}{fd}
\newcommand{\fC}{\f{C}}
\newcommand{\fD}{\f{D}}
\newcommand{\floor}[1]{\left\lfloor #1 \right\rfloor}
\newcommand{\fun}{\mapsto}
\newcommand{\norm}[1]{\vert \vert #1 \vert \vert}
\DeclareMathOperator{\FRel}{FRel}
\newcommand{\dgg}{{\f{g}^\bullet}}
\newcommand{\g}{\f{g}}
\DeclareMathOperator{\Gd}{Gd}
\DeclareMathOperator{\ev}{ev}
\DeclareMathOperator{\gldim}{gl.dim}
\let\H\relax %unassign \H
\DeclareMathOperator{\H}{H}
\DeclareMathOperator{\HH}{HH}
\DeclareMathOperator{\HC}{HC}
\DeclareMathOperator{\Hom}{Hom}
\DeclareMathOperator{\Id}{Id}
\DeclareMathOperator{\ID3}{ID_3}
\DeclareMathOperator{\im}{im}
\newcommand{\iso}{\stackrel{\sim}{\longrightarrow}}
\DeclareMathOperator{\Jac}{Jac}
\renewcommand{\k}{\Bbbk}
\DeclareMathOperator{\kNN}{kNN}
\newcommand{\K}{\mathbb{K}}
\DeclareMathOperator{\length}{length}
\DeclareMathOperator{\lrk}{\l.rk}
\newcommand{\leftperp}[1]{\leftidx{^\perp}{#1}}
\DeclareMathOperator{\LS}{LS}
\newcommand{\m}{\f{m}}
\DeclareMathOperator{\Mat}{Mat}
\DeclareMathOperator{\MCM}{MCM}
\DeclareMathOperator{\MC}{MC}
\DeclareMathOperator{\MCcal}{\mathcal{MC}}
\DeclareMathOperator{\MLE}{MLE}
\DeclareMathOperator{\Mod}{Mod}
\DeclareMathOperator{\Mor}{Mor}
\def\mod{\operatorname{mod}}
\newcommand{\mor}{\longrightarrow}
\newcommand{\T}{\mathscr{T}}
\renewcommand{\O}{\mathcal{O}}
\DeclareMathOperator{\NS}{NS}
\DeclareMathOperator{\Num}{Num}
\newcommand{\p}{\mathfrak{p}}
\DeclareMathOperator{\per}{per}
\DeclareMathOperator{\Ob}{Ob}
\DeclareMathOperator{\Perf}{Perf}
\DeclareMathOperator{\Pic}{Pic}
\DeclareMathOperator{\Proj}{Proj}
\DeclareMathOperator{\poly}{poly}
\DeclareMathOperator{\pred}{pred}
\DeclareMathOperator{\Qcoh}{Qcoh}
\DeclareMathOperator{\QGr}{QGr}
\DeclareMathOperator{\Rel}{Rel}
\DeclareMathOperator{\RHom}{RHom}
\renewcommand{\r}[1]{\mathcal{#1}}
\DeclareMathOperator{\rk}{rk}
\DeclareMathOperator{\rrk}{r.rk}
\DeclareMathOperator{\rsample}{RS}
\def\locExt{\mathscr{E}\mathit{xt}}
\newcommand{\sample}[1]{\textrm{S} #1}
\DeclareMathOperator{\SL}{SL}
\DeclareMathOperator{\RS}{RS}
\DeclareMathOperator{\SPS}{SPS}
\DeclareMathOperator{\sing}{sing}
\DeclareMathOperator{\sep}{sep}
\DeclareMathOperator{\Test}{Test}
\DeclareMathOperator{\test}{test}
\DeclareMathOperator{\Train}{Train}
\DeclareMathOperator{\train}{train}
\DeclareMathOperator{\true}{true}
\DeclareMathOperator{\vectorspan}{span}
\DeclareMathOperator{\Spec}{Spec}
\DeclareMathOperator{\Supp}{Supp}
\DeclareMathOperator{\SVM}{SVM}
\DeclareMathOperator{\SupVec}{SupVec}
\DeclareMathOperator{\Sym}{Sym}
\def\locHom{\operatorname {\mathscr{H}\mathit{om}}}
\DeclareMathOperator{\Tails}{Tails}
\def\ltr{\overset{L}{\tr}}
\renewcommand{\c}[1]{\mathcal{#1}}
\newcommand{\mono}{\xymatrix{{}\ar@{^{(}->}[r]&{}}}\DeclareMathOperator{\num}{num}
\DeclareMathOperator{\rad}{rad}
\DeclareMathOperator{\Tr}{Tr}
\DeclareMathOperator{\Tor}{Tor}
\DeclareMathOperator{\Tors}{Tors}
\newcommand{\tr}{\otimes}
\newcommand{\trps}[1]{ \leftidx{^{\operatorname{tr}}}{#1}}
\def\uRHom{\operatorname {R\mathcal{H}\mathit{om}}}
\DeclareMathOperator{\Var}{Var}
\newcommand{\w}{{\tt{w}}}
\newcommand{\Q}{\mathbb{Q}}
\DeclareMathOperator{\qis}{qis}
\DeclareMathOperator{\Qvr}{\mathcal{Q}}
\newcommand{\Z}{\mathbb{Z}}


%add dots between section and page in toc
\usepackage{tocloft}
\renewcommand{\cftsecleader}{\cftdotfill{\cftdotsep}}


%imake nterline a little larger
\linespread{1.28}

%change the dimensions of the margin
\usepackage{geometry}
 \geometry{
 a4paper,
 total={210mm,297mm},
 left=23mm,
 right=23mm,
 top=20mm,
 bottom=20mm,
 }
 
%changeqedsymbol
\renewcommand{\qedsymbol}{\CrossedBox}

%make the (sub)subsections run into the text, the 0.25 is the distance between the numbering and title
\titleformat{\subsection}[runin]{\normalfont\Large\bfseries}{\thesubsection.}{0.25em}{}
\titleformat{\subsubsection}[runin]{\normalfont\normalsize\bfseries}{\thesubsubsection.}{0.25em}{}

% remove indentation:
\setlength\parindent{0pt}

\begin{document}
\title{Analysis}

\maketitle	


\noindent\hrulefill
\tableofcontents
\noindent\hrulefill

\phantomsection
\label{section-phantom}

\section{Overview}
%reset maintheorem counter to use the corresponding letter
\renewcommand{\themaintheorem}{\Alph{maintheorem}}
\newtheorem{remark}[corollary]{Remark}
\newtheorem{proposition}[corollary]{Proposition}


\theoremstyle{definition}
\newtheorem{definition}[corollary]{Definition}
\newtheorem{definition*}{Definition}
\newtheorem{notation}{Notation}One major aspect of machine learning is the study of \emph{supervised learners}. \\
For these algorithms one is given a space of \emph{features} $\f{X}$ together with a space of \emph{labels} $\f{y}$ -the underlying idea being that to each feature $x \in \f{X}$, one can attach its correct label $y \in \f{y}$.\\ 
In order to do this, One first considers a \emph{hypothesis space} $\f{H}$ of functions $\f{X}\mor \f{y}$ which consist of all the ways one wishes to make a prediction, and collects a \emph{dataset} $\Delta \in \f{X}\times \f{y}$.\\
Given a dataset $\Delta$, one in turn defines a cost function $c: \f{H}\mor \d{R}$ which intuitively describes how accurate any chosen hypothesis $h \in \f{H}$ is. The appropriate $h_\Delta$ hypothesis will then be the one that minimizes this cost function.\\\\
As such the problem of finding minima for functions plays a key role in the field of machine learning.\\
Arguably, the most popular approach to this problem is to construct a sequence of hypotheses $h_i \in \f{H}$ which decreases the cost function $c(\Delta,h_i)$ at each step and hopefully converges to this global minimum. One particularly elegant way of constructing such a sequence is the method of gradient descent.\\
In this chapter, we investigate this method and prove that it is indeed descending and converging to a global minimum under certain circumstances. Along the way, we give a bound for the error at each iteration.
\section{Background and Notation}
In this context, the hypothesis space $\f{H}$ will be a Hilbert space, i.e. a real vector space $V$, together with an inner product $\bl{-}{-}$ such that the associated norm $\norm{x}\define \sqrt{\bl{x}{x}}$  is complete. For the reader not comfortable with the language of Hilbert spaces,  we assure him it is safe to assume that $V$ is finite-dimensional, in which case the Gram-Schmidt algorithm exhibits an orthonormal basis, and shows in particular that  $V$ is isomorphic to $\d{R}^n$, together with the usual inner product $\bl{x_1,\ldots , x_n}{y_1,\ldots y_n}\define \sum_i x_iy_i$.\\

\noindent We recall that given Hilbert spaces $V$ and $W$, we denote by $\Hom_\d{R}(V,W)$ the vector space of continuous and linear functions\footnote{in the case where $\dim V\neq \infty$, any linear map is automatically continuous so that the latter condition becomes superfluous}. In particular if $W=\d{R}$, we define $V^\star=\Hom_{\d{R}}(V,\d{R})$.  Moreover, for each element $x \in V$, we can consider $\bl{x}{-}:V\mor \d{R}$ which is clearly linear and continuous. This in turn defines a map $V\mor V^\star:x\mapsto \bl{x}{-}$ and it is a well known fact from the theory of Hilbert spaces that this map is an isomorphism.\\
\section{Differentiability}

\section{Subgradients}

We begin our analysis of gradient descent sequences by analyzing the role of convexity in minimizing functions. To this end, recall that:
\begin{definition}
A function $f:V\mor \d{R}$ is \emph{convex} if for any $x,y$ and $\alpha \in [0,1]$, we have
\[
f(\alpha x+(1-\alpha)y)\le \alpha f(x')+(1-\alpha)f(y)
\]	
\end{definition}

\noindent Our main reason for considering this property is the following characterization of the gradient:
\begin{theorem}\label{thm:gradient=subgradient}
Let $f:V\mor \d{R}$ be differentiable at $x$ and convex.\\
Then the gradient is the unique vector $\nabla f(x) \in V$ satisfying
\begin{equation}\label{ineq:subgradient}
f(y)-f(x)\ge \bl{\nabla f(x)}{ y-x}
\end{equation}
for all $y \in V$
\end{theorem}

\begin{proof}
\noindent First, we show that the statement is true for the vector $\nabla f(x)$:\\
By lemma \ref{lem:eq_defs_diff}, given $\epsilon>0$, for any $\vert \vert y-x \vert \vert <\delta $, we have
\[
\frac{\vert f(y)-f(x)-\Dir_x f(v) \vert }{\vert\vert y-x\vert\vert}= \frac{\vert f(y)-f(x)- \bl{\nabla f(x)}{y-x} \vert }{\vert\vert y-x\vert\vert}<\epsilon
\]
Removing the absolute values and reorganizing yields that
\[
f(y)-f(x)-\epsilon\vert \vert y-x\vert \vert \ge \bl{\nabla f(x)}{y-x}
\]
for all $y$ in some ball $B(x,\delta)$.\\
 We  now use the convexity of $f$ to show this in fact holds for \emph{any} $y \in V$. Since $\epsilon >0$ was chosen arbitrarily, the inequality (\ref{ineq:subgradient}) will follow immediately. \\
To this end, for any $y \in V$, consider the function
\begin{equation}\label{eq:gradient}
\psi(y)\define  f(y)-f(x)-\epsilon\vert\vert y-x\vert \vert - \bl{\nabla f(x)}{y-x}
\end{equation}
Equation (\ref{eq:gradient}) states that $\psi(y)\ge 0$ whenever $y \in B(x,\delta)$. We must now show that $\psi (y)\ge 0$ for arbitrary $y \in V$.\\
We leave it to the reader to verify that $\psi$ itself is convex and note that it is trivial that $\psi(x)=0$.\\ 
Now, pick $\alpha< \frac{\delta}{\vert \vert y-x\vert \vert }$ small enough so that $\alpha y+(1-\alpha )x \in B(x,\delta)$. Then by convexity, we have:
\[
\alpha \psi(y)=\alpha \psi(y)+(1-\alpha)\psi(x)\ge \psi\big(\alpha y+(1-\alpha)x\big)\ge 0
\]
It follows in particular that $\psi(y)\ge 0$, hence the existence.\\
To show that $\nabla f(x)$ is the unique vector satisfying the inequality (\ref{ineq:subgradient}), we assume  that $v \in V$ satisfies
\[
f(y)-f(x)\ge \bl{v}{ y-x}
\]
And conclude that
\[
\frac{\bl{ v-\nabla f(x)}{y-x} }{\vert \vert y-x \vert\vert }\le \frac{f(y)-f(x)-\bl{\nabla f(x)}{y-x} }{\vert \vert y-x \vert\vert }\le \frac{\vert f(y)-f(x)-\bl{\nabla f(x)}{(y-x)}\vert }{\vert \vert y-x\vert \vert}
\]
The differentiability of$f$ at $x$ now implies that for any $\epsilon>0$
\[
\frac{\bl{ v-\nabla f(x)}{y-x} }{\vert \vert y-x \vert\vert }<\epsilon
\]
for all $y$ in some ball $B(x,\delta)$. Using the cosine rule we conclude  that \[\frac{\bl{ v-\nabla f(x)}{y-x}}{\vert \vert y-x\vert \vert}=\vert \vert v-\nabla f(x)\vert \vert  \cdot \cos(\theta)<\epsilon\]

Where $\theta$ denotes the angle between $v$ and $\nabla f(x)$ which immediately implies that $v=\nabla f(x)$, as $\epsilon$ was chosen arbitrarily.
\end{proof}

\noindent This characterization allows us to introduce gradients in the more general setting of convex, non-differentiable functions:
\begin{definition}
Let $f:V\mor \d{R}$ be a convex function. The \emph{subgradient} at $x$ is the set $\partial f(x)$ of all vectors $v \in V$ such that 
\[
f(y)-f(x)\ge \bl{v}{ y-x}
\]	
for any $y \in V$
\end{definition}

\noindent The above theorem shows that in the case where $f$ is differentiable, we have
\[
\partial f(x)=\{\nabla f(x)\}
\]
A crucial theorem in the theory of convex optimization is:
\begin{theorem}
Let $f:V \mor \d{R}$ be convex, then the subgradient is a nonempty set.	
\end{theorem}

\noindent The interpretation of the subgradient is rather powerful, as the next two results are easily proven:
\begin{lemma}\label{lem:stationary_point=global_minimum}
Let $f:V\mor \d{R}$ be convex and differentiable everywhere. Then
\begin{itemize}
\item any local minimum is a global one
\item Any stationary point is an either a global minimum or maximum	
\end{itemize}
\end{lemma}

\begin{proof}
To prove the first point, we use a standard argument already used in \ref{thm:gradient=subgradient}:
If $f(y)\ge f(x)$ for all $y$ in some ball $B(x,\delta)$, then we pick $\alpha<\frac{\delta}{\vert \vert y-x \vert\vert }$ small enough so that $\alpha y+(1-\alpha)x \in B(x,\delta)$, then using convexity, we obtain:
\[
\alpha f(y)+(1-\alpha)f(x)\ge f(\alpha y+(1-\alpha)x)\ge f(x)
\]
Which immediately proves the first claim.\\
\noindent For the second, if $\nabla f(x)=0$, then using the fact that $\nabla f(x)$ is the subgradient, for any $y \in V$, we have
\[
f(y)\ge f(x)-\nabla f(x)(y-x)=f(x)
\]
\end{proof}

We now  go on to prove an important inequality in a certain setting, which in some sense provides a converse to inequality (\ref{ineq:subgradient})

\begin{lemma}
Assume $f$ is convex and differentiable everywhere. Moreover, assume the gradient satisfies the Lipschitz condition:
\[
\vert  \nabla f(y)-\nabla f(x) \vert \le K\cdot \norm{y-x}
\]	
Then for any $x,y \in V$, we have
\[
f(x)+\bl{\nabla f(x)}{(y-x)} \le f(y)\le f(x)+\bl{\nabla f(x)}{y-x} + K \cdot \norm{y-x}^2
\]
\end{lemma}

\begin{proof}
The left hand side of the inequality is simply the subgradient inequality (\ref{ineq:subgradient}).\\
To prove the right-hand side we once again use the subradient inequality (\ref{ineq:subgradient}) to compute:
\begin{align*}
f(y)-f(x)-\bl{\nabla f(x)}{y-x}&\le f(y)-\bigg(f(y)+\bl{\nabla f(y)}{x-y}\bigg)-\bl{\nabla f(x)}{y-x}\\
&=\bl{\nabla f(y)-\nabla f(x)}{y-x}\\
&\le \vert \vert \nabla f(y)-\nabla f(x)\vert \vert \cdot \vert \vert y-x\vert \vert\\
&\le K\cdot \vert \vert y-x\vert \vert ^2
\end{align*}

\end{proof}

\section{Gradient descent sequences}

Given a function $f:V\mor W$ and a point $x\in V$, we say that $f$ is \emph{differentiable} at $x$ if for each $v \in V$, the \emph{directional derivative}

\[
\Dir_xf(v)\define \lim_{\lambda \to 0}\frac{f(x+\lambda v)-f(x)}{\lambda}
\]

\noindent is defined for any $v \in V$  and  that  $\Dir_xf(-):V\mor W$ is linear and continuous. A classical result in analysis is the folllowing:
\begin{lemma}\label{lem:eq_defs_diff}
A function is differentiable at $x$ iff there exists a linear map $L_x \in \Hom_\d{R}(V,W)$ such that
\[
\lim_{\vert \vert v \vert \vert \mor 0}\frac{\vert \vert f(x+v)-f(x)-L_x(v)\vert \vert}{\vert \vert v\vert \vert}=0
\]
In which case $L_x=\Dir_xf$
\end{lemma}
Furthermore, if $W=\d{R}$, we can make the following observation:

\begin{lemma}
Assume that $f:V\mor \d{R}$ is differentiable. Then there exists a unique element $\nabla f(x) \in V$ such that 
\[
\Dir_xf(-)=\bl{\nabla f(x)}{-}
\]
\end{lemma}
 \begin{proof}
 Since $\Dir_xf$ lies in $V^\star$ by the definition of differentiability and since the map \[V\mor V^*:x\mor \bl{x}{-} \]	is an isomorphism, the result follows.
 \end{proof}

\begin{definition}
Let $f: V\mor \d{R}$ be differentiable. The function \[\nabla f:  V\mor V: x\fun \nabla f(x)\] is the \emph{gradient} of $f$.
\end{definition}
 The gradient has another interesting interpretation. To this end, for any $x$, we define the direction $\overline{x}$ of $x$ as the class under the equivalence relation $x\sim y \iff \d{R}^+x=\d{R}^+y$. 
\begin{remark}
The direction of $\nabla f(x)$ yields the minimal directional derivative. Indeed, for any other direction $\overline{Y}$ represented by a unit vector $y$, we compute
\[
 \Dir_xf(y) =  \bl{\nabla f(x)}{ y} =\vert \vert \nabla f(x)\vert \vert \cdot \vert \vert y \vert \vert \cos(\theta)=\vert \vert \nabla f(x) \vert \vert \cdot \cos \theta
\] 
where $\theta$ denotes the angle between $x$ and $y$. This minimum is attained when $\cos(\theta)=-1$, in other words for any vector of direction $-\nabla f(x)$, by a simple application of the Cauchy-Schwarz theorem 
\end{remark}

\noindent This observation motivates gradient descent.\\
 Indeed, as mentioned in the Overview, our goal is to minimize the function $f$. One approach would be to begin with any choice of starting point $x_0$ and subsequently build a sequence in such a way that the direction of any vector $x_{i+1}-x_i$ is $\nabla f(x_i)$, the minimal directional derivative.
\begin{definition}
A gradient descent sequence is a sequence $(x_i)_{i \in \d{N}} \in V$ such that the direction of two subsequent elements satisfies
\[
\overline{x_{i+1}-x_i}=\overline{\nabla f(x_i))}
\]
In this case, we necessarily have \[x_{i+1}=x_i-\lambda_i\nabla f(x_i)\]
\noindent We call $(\lambda_i)_i$ the sequence of \emph{learning rates.}
\end{definition}
\noindent This chapter is concerned with understanding when gradient descent sequences converge. The main result here being that in the case where $f$ is differentiable and convex, $\delta f$ is a Lipschitz function with constant $K$ and the learning rates are constant such that $\lambda \le \frac{1}{K}$, we can describe the convergence rather well.

\section{Subgradients}
We begin our analysis of gradient descent sequences by analyzing the role of convexity in minimizing functions. To this end, recall that:
\begin{definition}
A function $f:V\mor \d{R}$ is \emph{convex} if for any $x,y$ and $\alpha \in [0,1]$, we have
\[
f(\alpha x+(1-\alpha)y)\le \alpha f(x')+(1-\alpha)f(y)
\]	
\end{definition}

\noindent Our main reason for considering this property is the following characterization of the gradient:
\begin{theorem}\label{thm:gradient=subgradient}
Let $f:V\mor \d{R}$ be differentiable at $x$ and convex.\\
Then the gradient is the only vector $\nabla f(x) \in V$ satisfying
\begin{equation}\label{ineq:subgradient}
f(y)-f(x)\ge \bl{\nabla f(x)}{ y-x}
\end{equation}
for all $y \in V$
\end{theorem}

\begin{proof}
\noindent First, we show that the statement is true for the vector $\nabla f(x)$:\\
By lemma \ref{lem:eq_defs_diff}, given $\epsilon>0$, for any $\vert \vert y-x \vert \vert <\delta $, we have
\[
\frac{\vert f(y)-f(x)-\Dir_x f(v) \vert }{\vert\vert y-x\vert\vert}= \frac{\vert f(y)-f(x)- \bl{\nabla f(x)}{y-x} \vert }{\vert\vert y-x\vert\vert}<\epsilon
\]
Removing the absolute values and reorganizing yields that
\[
f(y)-f(x)-\epsilon\vert \vert y-x\vert \vert \ge \bl{\nabla f(x)}{y-x}
\]
for all $y$ in some ball $B(x,\delta)$.\\
 We  now use the convexity of $f$ to show this in fact holds for \emph{any} $y \in V$. Since $\epsilon >0$ was chosen arbitrarily, the inequality (\ref{ineq:subgradient}) will follow immediately. \\
To this end, for any $y \in V$, consider the function
\begin{equation}\label{eq:gradient}
\psi(y)\define  f(y)-f(x)-\epsilon\vert\vert y-x\vert \vert - \bl{\nabla f(x)}{y-x}
\end{equation}
Equation (\ref{eq:gradient}) states that $\psi(y)\ge 0$ whenever $y \in B(x,\delta)$. We must now show that $\psi (y)\ge 0$ for arbitrary $y \in V$.\\
We leave it to the reader to verify that $\psi$ itself is convex and note that it is trivial that $\psi(x)=0$.\\ 
Now, pick $\alpha< \frac{\delta}{\vert \vert y-x\vert \vert }$ small enough so that $\alpha y+(1-\alpha )x \in B(x,\delta)$. Then by convexity, we have:
\[
\alpha \psi(y)=\alpha \psi(y)+(1-\alpha)\psi(x)\ge \psi\big(\alpha y+(1-\alpha)x\big)\ge 0
\]
It follows in particular that $\psi(y)\ge 0$, hence the existence.\\
To show that $\nabla f(x)$ is the unique vector satisfying the inequality (\ref{ineq:subgradient}), we assume  that $v \in V$ satisfies
\[
f(y)-f(x)\ge \bl{v}{ y-x}
\]
And conclude that
\[
\frac{\bl{ v-\nabla f(x)}{y-x} }{\vert \vert y-x \vert\vert }\le \frac{f(y)-f(x)-\bl{\nabla f(x)}{y-x} }{\vert \vert y-x \vert\vert }\le \frac{\vert f(y)-f(x)-\bl{\nabla f(x)}{(y-x)}\vert }{\vert \vert y-x\vert \vert}
\]
The differentiability of$f$ at $x$ now implies that for any $\epsilon>0$
\[
\frac{\bl{ v-\nabla f(x)}{y-x} }{\vert \vert y-x \vert\vert }<\epsilon
\]
for all $y$ in some ball $B(x,\delta)$. Using the cosine rule we conclude  that \[\frac{\bl{ v-\nabla f(x)}{y-x}}{\vert \vert y-x\vert \vert}=\vert \vert v-\nabla f(x)\vert \vert  \cdot \cos(\theta)<\epsilon\]

Where $\theta$ denotes the angle between $v$ and $\nabla f(x)$ which immediately implies that $v=\nabla f(x)$, as $\epsilon$ was chosen arbitrarily.
\end{proof}

\noindent This characterization allows us to introduce gradients in the more general setting of convex, non-differentiable functions:
\begin{definition}
Let $f:V\mor \d{R}$ be a convex function. The \emph{subgradient} at $x$ is the set $\partial f(x)$ of all vectors $v \in V$ such that 
\[
f(y)-f(x)\ge \bl{v}{ y-x}
\]	
for any $y \in V$
\end{definition}

\noindent The above theorem shows that in the case where $f$ is differentiable, we have
\[
\partial f(x)=\{\nabla f(x)\}
\]
A crucial theorem in the theory of convex optimization is:
\begin{theorem}
Let $f:V \mor \d{R}$ be convex, then the subgradient is a nonempty set.	
\end{theorem}

\noindent The interpretation of the subgradient is rather powerful, as the next two results are easily proven:
\begin{lemma}\label{lem:stationary_point=global_minimum}
Let $f:V\mor \d{R}$ be convex and differentiable everywhere. Then
\begin{itemize}
\item any local minimum is a global one
\item Any stationary point is an either a global minimum or maximum	
\end{itemize}
\end{lemma}

\begin{proof}
To prove the first point, we use a standard argument already used in \ref{thm:gradient=subgradient}:
If $f(y)\ge f(x)$ for all $y$ in some ball $B(x,\delta)$, then we pick $\alpha<\frac{\delta}{\vert \vert y-x \vert\vert }$ small enough so that $\alpha y+(1-\alpha)x \in B(x,\delta)$, then using convexity, we obtain:
\[
\alpha f(y)+(1-\alpha)f(x)\ge f(\alpha y+(1-\alpha)x)\ge f(x)
\]
Which immediately proves the first claim.\\
\noindent For the second, if $\nabla f(x)=0$, then using the fact that $\nabla f(x)$ is the subgradient, for any $y \in V$, we have
\[
f(y)\ge f(x)-\nabla f(x)(y-x)=f(x)
\]
\end{proof}

We now  go on to prove an important inequality in a certain setting, which in some sense provides a converse to inequality (\ref{ineq:subgradient})

\begin{lemma}
Assume $f$ is convex and differentiable everywhere. Moreover, assume the gradient satisfies the Lipschitz condition:
\[
\vert  \nabla f(y)-\nabla f(x) \vert \le K\cdot \vert \vert y-x\vert \vert 
\]	
Then for any $x,y \in V$, we have
\[
f(x)+\nabla f(x)(y-x) \le f(y)\le f(x)+\bl{\nabla f(x)}{y-x} + K \cdot \vert \vert y-x\vert \vert ^2
\]
\end{lemma}

\begin{proof}
The left hand side of the inequality is simply the subgradient inequality (\ref{ineq:subgradient}).\\
To prove the right-hand side we once again use the subradient inequality (\ref{ineq:subgradient}) to compute:
\begin{align*}
f(y)-f(x)-\bl{\nabla f(x)}{y-x}&\le f(y)-\bigg(f(y)+\bl{\nabla f(y)}{x-y}\bigg)-\bl{\nabla f(x)}{y-x}\\
&=\bl{\nabla f(y)-\nabla f(x)}{y-x}\\
&\le \vert \vert \nabla f(y)-\nabla f(x)\vert \vert \cdot \vert \vert y-x\vert \vert\\
&\le K\cdot \vert \vert y-x\vert \vert ^2
\end{align*}

\end{proof}

This inequality forms the key to the gradient descent convergence theorem.
\begin{theorem}[convergence of GD]
Assume $f$ is differentiable and convex, and  that $\nabla f$ is Lipschitz with constant $K$. Assume $\lambda$ is a constant sequence of learning rates such that 
\[\lambda <\frac{1}{K}\]. Assume that $f$ reaches a minimum at $x_\mu$. Then
\begin{itemize}
\item the sequence $\vert \vert x_{i+1}-x_i\vert \vert$ is square summable and converges to $x_\mu$
\item  For $f^\mu\define f(x_\mu)$. We have the following bound:
\[
\vert f(x_i)-f^\mu \vert \le \frac{\vert \vert x_0-x_\mu \vert \vert }{ 2\lambda \cdot K}
\]	
\end{itemize}
\end{theorem}

\begin{proof}
Let $x \in V$, $\lambda \in \d{R}$ and define $x^+\define x-\lambda \nabla f(x)$.\\
\noindent then we compute:
\begin{align*}
	f(x^+)& \le f(x)+\bl{\nabla f(x)}{x^+-x}+K\vert \vert x^+-x\vert \vert^2\\
	& = f(x)-\lambda \vert \vert \nabla f(x)\vert \vert^2 +K\lambda^2\vert \vert \nabla f(x)\vert \vert^2\\
	&= f(x)-\big(\lambda- K\lambda^2\big)\cdot \vert \vert \nabla f(x)\vert \vert^2
\end{align*}
From this last in equality we can draw a few conclusions:
\begin{itemize}
\item	First, since $\lambda \le \frac{1}{K}$, we have $\lambda-K\lambda^2\ge 0$. In particular: 
\[f(x^+)\le f(x)\]
\item Next, rewriting, we see that
\[
\norm{\nabla f(x)}^2\le \frac{1}{K\lambda^2-\lambda}\bigg(f(x)-f(x^+)\bigg)
\]
Implying that for a gradient sequence $(x_i)_i$ (using telescopic sums), we have
\[
\sum_i^n \norm{x_{i+1}-x_i}^2 =\lambda^2 \sum_i^n \norm{\nabla f(x_i)}^2\le \frac{\lambda^2}{K\lambda^2-\lambda}\vert f(x_0)-f(x_n)\vert\le \frac{\lambda^2}{K\lambda-\lambda^2}\vert f(x_0)-f^\mu \vert  
\]
We conclude that the sequence $\norm{x_{i+1}-x_i}$ is square summable. Moreover, since the gradient is Lipschitz, it is continuous in particular. Hence
\[
0=\lim \nabla f(x_i)=\nabla f (\lim x_i))
\]
meaning that $\nabla f\big( \lim(x_i)\big)$ is a stationary point, hence the global minimum  $x_\mu$ by Lemma \ref{lem:stationary_point=global_minimum}
\end{itemize}
\noindent To prove the second bound, we first apply the subgradient inequality (\ref{ineq:subgradient}) to $x$: \[f(x)\le f(x_\mu)+\bl{\nabla f(x)}{x-x_\mu}\] to obtain
\begin{align*}
f(x^+)&\le \bigg( f(x_\mu)+\bl{\nabla f(x)}{x-x_\mu}\bigg)+(K\lambda^2-\lambda)\cdot \vert \vert \nabla f(x)\vert \vert^2	
\end{align*}
Now since $K$ and $\lambda$ are positive, we conclude $K\lambda^2-\lambda \ge \frac{\lambda}{2}$, so that
\begin{align*}
f(x^+)-f(x_\mu)&\le  f(x_\mu)+\bl{\nabla f(x)}{x-x_\mu}-(K\lambda^2-\lambda)\cdot \vert \vert \nabla f(x)\vert \vert^2	\\ 
 & \le \frac{1}{2\lambda}\bigg(2\lambda \bl{\nabla f(x)}{x -x_\mu}-\lambda^2\vert \vert \nabla f(x)\vert \vert^2 \bigg)\\
 & = \frac{1}{2\lambda}\bigg(\vert \vert x - x_\mu\vert \vert^2 -\big(\vert \vert x-x_\mu \vert \vert^2 -2\lambda \bl{\nabla f(x)}{x -x_\mu}+\lambda^2\vert \vert \nabla f(x)\vert \vert^2 \big)\bigg)\\
 &=\frac{1}{2\lambda}\bigg(\vert \vert x-x_\mu\vert \vert^2-\vert \vert x-\lambda \nabla f(x)-x_\mu \vert \vert^2 \bigg)\\
 &=\frac{1}{2\lambda}\bigg(\vert \vert x-x_\mu\vert \vert^2-\vert \vert x^+-x_\mu \vert \vert^2 \bigg)\\
 \end{align*}
We return to our gradient descent sequence $(x_i)_i$. Then
\begin{align*}
\sum_i f(x_{i+1})-f^\mu	& \le \sum_i \frac{1}{2\lambda}\bigg(\vert \vert x_i-x_\mu\vert \vert^2-\vert \vert x_{i+i}-x_\mu \vert \vert^2 \bigg)
&= \frac{1}{2\lambda}\bigg(\vert \vert x_0-x_\mu\vert \vert^2-\vert \vert x_{i+1}-x_\mu \vert \vert^2 \bigg)
&\le \frac{1}{2\lambda}\vert \vert x_0-x\vert\vert^2
\end{align*}
Moreover, since $f$ decreases with each iteration by the above, it follows that 
\[k\cdot \big(f(x_n)-f^\mu\big)\le \sum^n_i f(x_i)-f^\mu
\], implying that
\[
f(x_{n})-f^\mu\le \frac{\norm{x_0-x^*}^2}{2\lambda k}
\]
\end{proof}


\end{document}