\documentclass{article}
\usepackage{amsmath,amsthm,amsfonts,amssymb,tikz,scalerel,mathrsfs,marvosym,titlesec,leftidx}

%for diagrams
\usepackage[all]{xy}

\usepackage{fancyhdr}
\usepackage{blkarray}
\usepackage{calrsfs}



%for referencing accross documents
\usepackage{xr-hyper}

\externaldocument[probability-]{probability}



%change the default font to bodoni
\usepackage[default]{gfsbodoni}
\usepackage[T1]{fontenc}

% Declare calligraphic font:
\DeclareMathAlphabet{\pr}{OMS}{zplm}{m}{n}


%start at section 0 for the overview: 
\setcounter{section}{-1}


%environments
%
\theoremstyle{plain}
\newtheorem{corollary}{Corollary}[section]
\newtheorem{convention}[corollary]{Convention}
\newtheorem{example}[corollary]{Example}
\newtheorem*{theorem*}{theorem}
\newtheorem{theorem}[corollary]{Theorem}
\newtheorem{lemma}[corollary]{Lemma}
\newtheorem*{lemma*}{Lemma}
\newtheorem{maintheorem}{Theorem}
%reset maintheorem counter to use the corresponding letter
\renewcommand{\themaintheorem}{\Alph{maintheorem}}
\newtheorem{remark}[corollary]{Remark}
\newtheorem{proposition}[corollary]{Proposition}


\theoremstyle{definition}
\newtheorem{definition}[corollary]{Definition}
\newtheorem{definition*}{Definition}
\newtheorem{notation}{Notation}

%macros
%
\newcommand{\A}{\mathcal{A}}
\DeclareMathOperator{\avg}{avg}
\newcommand{\abs}[1]{\vert #1 \vert }
\DeclareMathOperator{\ad}{ad}
\DeclareMathOperator{\Aff}{aff}
\DeclareMathOperator{\arginf}{arginf}
\DeclareMathOperator{\argmin}{argmin}
\DeclareMathOperator{\argsup}{argsup}
\DeclareMathOperator{\argmax}{argmax}
\DeclareMathOperator{\Aut}{Aut}
\renewcommand{\b}{\bullet}
\newcommand{\bl}[2]{\left\langle #1\,,#2\, \right\rangle}
\newcommand{\bp}{\mathbf{\Pi}}
\def\BVm{\operatorname{BV}_-}
\DeclareMathOperator{\BiMod}{BiMod}
\DeclareMathOperator{\bimod}{bimod}
\DeclareMathOperator{\card}{card}
\newcommand{\C}{{\mathcal{C}}}
\DeclareMathOperator{\CC}{CC}
\def\cd{\operatorname {cd}}
\DeclareMathOperator{\ch}{ch}
\DeclareMathOperator{\CH}{CH}
\DeclareMathOperator{\Conv}{Conv}
\DeclareMathOperator{\co}{co}
\DeclareMathOperator{\coh}{coh}
\DeclareMathOperator{\cone}{cone}
\DeclareMathOperator{\coker}{coker}
\renewcommand{\d}[1]{\mathbb{#1}}
\newcommand{\D}{{\mathscr{D}}}
\DeclareMathOperator{\der}{R}
\DeclareMathOperator{\Def}{Def}
\DeclareMathOperator{\Dir}{Dir}
\newcommand{\define}{\stackrel{\operatorname{def}}{=}}
\let\div\relax %unassign \div
\DeclareMathOperator{\div}{div}
\DeclareMathOperator{\divu}{div}
\newcommand{\draw}[1]{\widetilde{\textrm{S}} #1}
\newcommand{\ds}{\oplus}
\newcommand{\DS}{\bigoplus}
\newcommand{\epi}{\xymatrix{{}\ar@{->>}[r]&{}}}
\DeclareMathOperator{\End}{End}
\DeclareMathOperator{\Entropy}{Entropy}
\DeclareMathOperator{\Ext}{Ext}
\newcommand{\f}[1]{\mathfrak{#1}}
\DeclareMathOperator{\fd}{fd}
\newcommand{\fC}{\f{C}}
\newcommand{\fD}{\f{D}}
\newcommand{\floor}[1]{\left\lfloor #1 \right\rfloor}
\newcommand{\fun}{\mapsto}
\newcommand{\norm}[1]{\vert \vert #1 \vert \vert}
\DeclareMathOperator{\FRel}{FRel}
\newcommand{\dgg}{{\f{g}^\bullet}}
\newcommand{\g}{\f{g}}
\DeclareMathOperator{\Gd}{Gd}
\DeclareMathOperator{\ev}{ev}
\DeclareMathOperator{\gldim}{gl.dim}
\let\H\relax %unassign \H
\DeclareMathOperator{\H}{H}
\DeclareMathOperator{\HH}{HH}
\DeclareMathOperator{\HC}{HC}
\DeclareMathOperator{\Hom}{Hom}
\DeclareMathOperator{\Id}{Id}
\DeclareMathOperator{\ID3}{ID_3}
\DeclareMathOperator{\im}{im}
\newcommand{\iso}{\stackrel{\sim}{\longrightarrow}}
\DeclareMathOperator{\Jac}{Jac}
\renewcommand{\k}{\Bbbk}
\DeclareMathOperator{\kNN}{kNN}
\newcommand{\K}{\mathbb{K}}
\DeclareMathOperator{\length}{length}
\DeclareMathOperator{\lrk}{\l.rk}
\newcommand{\leftperp}[1]{\leftidx{^\perp}{#1}}
\DeclareMathOperator{\LS}{LS}
\newcommand{\m}{\f{m}}
\DeclareMathOperator{\Mat}{Mat}
\DeclareMathOperator{\MCM}{MCM}
\DeclareMathOperator{\MC}{MC}
\DeclareMathOperator{\MCcal}{\mathcal{MC}}
\DeclareMathOperator{\MLE}{MLE}
\DeclareMathOperator{\Mod}{Mod}
\DeclareMathOperator{\Mor}{Mor}
\def\mod{\operatorname{mod}}
\newcommand{\mor}{\longrightarrow}
\newcommand{\T}{\mathscr{T}}
\renewcommand{\O}{\mathcal{O}}
\DeclareMathOperator{\NS}{NS}
\DeclareMathOperator{\Num}{Num}
\newcommand{\p}{\mathfrak{p}}
\DeclareMathOperator{\per}{per}
\DeclareMathOperator{\Ob}{Ob}
\DeclareMathOperator{\Perf}{Perf}
\DeclareMathOperator{\Pic}{Pic}
\DeclareMathOperator{\Proj}{Proj}
\DeclareMathOperator{\poly}{poly}
\DeclareMathOperator{\pred}{pred}
\DeclareMathOperator{\Qcoh}{Qcoh}
\DeclareMathOperator{\QGr}{QGr}
\DeclareMathOperator{\Rel}{Rel}
\DeclareMathOperator{\RHom}{RHom}
\renewcommand{\r}[1]{\mathcal{#1}}
\DeclareMathOperator{\rk}{rk}
\DeclareMathOperator{\rrk}{r.rk}
\DeclareMathOperator{\rsample}{RS}
\def\locExt{\mathscr{E}\mathit{xt}}
\newcommand{\sample}[1]{\textrm{S} #1}
\DeclareMathOperator{\SL}{SL}
\DeclareMathOperator{\RS}{RS}
\DeclareMathOperator{\SPS}{SPS}
\DeclareMathOperator{\sing}{sing}
\DeclareMathOperator{\sep}{sep}
\DeclareMathOperator{\Test}{Test}
\DeclareMathOperator{\test}{test}
\DeclareMathOperator{\Train}{Train}
\DeclareMathOperator{\train}{train}
\DeclareMathOperator{\true}{true}
\DeclareMathOperator{\vectorspan}{span}
\DeclareMathOperator{\Spec}{Spec}
\DeclareMathOperator{\Supp}{Supp}
\DeclareMathOperator{\SVM}{SVM}
\DeclareMathOperator{\SupVec}{SupVec}
\DeclareMathOperator{\Sym}{Sym}
\def\locHom{\operatorname {\mathscr{H}\mathit{om}}}
\DeclareMathOperator{\Tails}{Tails}
\def\ltr{\overset{L}{\tr}}
\renewcommand{\c}[1]{\mathcal{#1}}
\newcommand{\mono}{\xymatrix{{}\ar@{^{(}->}[r]&{}}}\DeclareMathOperator{\num}{num}
\DeclareMathOperator{\rad}{rad}
\DeclareMathOperator{\Tr}{Tr}
\DeclareMathOperator{\Tor}{Tor}
\DeclareMathOperator{\Tors}{Tors}
\newcommand{\tr}{\otimes}
\newcommand{\trps}[1]{ \leftidx{^{\operatorname{tr}}}{#1}}
\def\uRHom{\operatorname {R\mathcal{H}\mathit{om}}}
\DeclareMathOperator{\Var}{Var}
\newcommand{\w}{{\tt{w}}}
\newcommand{\Q}{\mathbb{Q}}
\DeclareMathOperator{\qis}{qis}
\DeclareMathOperator{\Qvr}{\mathcal{Q}}
\newcommand{\Z}{\mathbb{Z}}


%add dots between section and page in toc
\usepackage{tocloft}
\renewcommand{\cftsecleader}{\cftdotfill{\cftdotsep}}


%imake nterline a little larger
\linespread{1.28}

%change the dimensions of the margin
\usepackage{geometry}
 \geometry{
 a4paper,
 total={210mm,297mm},
 left=23mm,
 right=23mm,
 top=20mm,
 bottom=20mm,
 }
 
%changeqedsymbol
\renewcommand{\qedsymbol}{\CrossedBox}

%make the (sub)subsections run into the text, the 0.25 is the distance between the numbering and title
\titleformat{\subsection}[runin]{\normalfont\Large\bfseries}{\thesubsection.}{0.25em}{}
\titleformat{\subsubsection}[runin]{\normalfont\normalsize\bfseries}{\thesubsubsection.}{0.25em}{}

% remove indentation:
\setlength\parindent{0pt}

\begin{document}
\title{SUPPORT VECTOR MACHINES}

\maketitle	


\noindent\hrulefill
\tableofcontents
\noindent\hrulefill

\phantomsection
\label{section-phantom}

\section{Overview}

\section{Overview}
\begin{convention}\label{conv:stats_univ-prob}
Throughout, we will fix a probability space $(U,\r{U},\mu)$
\end{convention}
\section{Statistical models}
\begin{definition}
Let $(\Theta, \r{O})$ and $(\Omega,\r{F})$ be measurable spaces.\\
A Markov kernel (denoted $\Theta\implies \Omega$) is a map

\[
\Sigma: \Omega \times \r{F}\mor \d{R}
\]
Where for each $\theta \in \Theta$ the function $\Sigma(\theta,-)$ is a probability measure on $\Omega$ and for each $A \in\r{F}$, the function $\sigma(-,A)$ is $\r{O}$-measurable.\\
If each probability measure $\Sigma(\theta,-)$ is dominated by the measure $\lambda$ (conv. \ref{conv:stats_univ-prob}), then we write $\Sigma(\theta,-)\define P_\theta$ and call it a \emph{statistical model}
\end{definition}

\begin{example}
Let $(\Omega, \r{F})$ be a measurable space and consider the  space  of samples (def. \ref{mb:prob:def:sampling}) given by $S\Omega\define \coprod_{in \in\d{N}}\Omega^i$.\\
Then we can define a statistical model $\Sigma: S\Omega \implies \Omega$ through
\[
\sigma\big((x_1,\ldots, x_n), A\big)=\frac{1}{n}\sum \card\big\{1_{A}(x_i)\big\}
\]
This is the descriptive statistical model of $(\Omega, \r{F})$\end{example}
\subsection{the category Stat$(\Theta)$}
The collection of statistical models on  the parameter space $\Theta$ naturally forms a category by defining a morphism as follows:



\subsubsection{the pullback functor}

\begin{definition}
Let $p:\Theta'\mor \Theta$ be any map (we intuitively think of $p$ as focussing on a  property of $\Theta$).\\
Let $\Sigma: \Theta\implies \Omega$ be a statistical model.\\
Then we can define the push-forward of $\Sigma$ along $p$, denoted $p^*\Sigma$ as 
\[
p^*\Sigma: \Theta\times \r{F}\mor \d{R}: \big(\omega,A)\fun \Sigma\big(p(\omega), A\big)
\]
\end{definition}



\section{Statistics}

\noindent An important aspect of statistical models is that given a sample $x \in S\Omega$ we'd like to find a way to asociate a parameter $\theta \in \Theta$ (a process known as \emph{estimation}).\\
To this end, we  recall definition \ref{prob:def.sampling} and introduce an \emph{estimation space} as a set $E$ and define:

\begin{definition}\label{stats:def.estimators}
An estimator for the statistical model $\Sigma$ is a sequence of functions
\begin{displaymath}
\xymatrix{
S\Omega\ar[r]_{e} & E\ar[r]_p  & \Theta \cup \{\infty\}
}
\end{displaymath}
such that the image of the composition lies in $\Theta$.\\
Note that for any $\omega \in \Omega$, we immediately obtain a map $RS\Omega\mor \Theta$ from the random sample space given by
\begin{displaymath}
\xymatrix{
S\ar[r]^e\Omega & E\ar[d]^p\\
RS\ar[u]^{\ev_{\omega}}\ar[r]_{e_R}\Omega & \Theta\cup\{ \infty \}
}
\end{displaymath}
\end{definition}

\section{Bayesian models}


\begin{definition}
Let $(\Theta,\r{O})$ and $(\Omega,\r{F})$ be measurable spaces.
A Bayesian model is a probability measure $\Pi$ on the measurable space $(\Theta\times \Omega, \r{O}\times\r{F})$
\end{definition}

\part{statistics}



If a statistical model is dominated, one can consider the following map
\[
\MLE: \Sigma\mor \r{P}(\Omega):x \fun \argsup_\theta f(\theta,x)
\]
We call $L \in \Hom(\Sigma,\Omega$ a maximum likelihood estimator if $L(\theta)\in MLE$.
\section{Bayesian Models}



An important subset of statistical models are so-called Bayesian models in which one considers extra structure which lifts $(\Theta,\r{O})$ to a probability space itself. Assume we are given a probability measure $P$ on $(\Theta,\r{O})$, then together with the statistical model $P_\r{O}$, one can form the probability $\Pi\define P\rtimes P_\theta$ following theorem \ref{thm:integratemarkovkernel}  and since one can recover $P$ and $P_\theta$ from $\Pi$, it makes sense to define:
\begin{definition}\label{def:stochasticmodel}
A stochastic statistical model is a probability measure $\Pi$ on $\Theta\times \Sigma, \r{O}\times \r{S})$ which decomposes into $P\rtimes P_\r{O}$
\end{definition}

One can go one step further and demand this definition be symmetric:

\begin{definition}\label{def:bayesianmodel}
A Bayesian model  $\Pi$ is a probability measure on $\Theta\times \Sigma, \r{O}\times \r{S})$ which decomposes into $\mu \rtimes \nu_\r{O}$ and $\nu\rtimes \mu_\r{S}$	
\end{definition}

Before we go on, we wish to introduce a bit of abuse of notation in two steps. First we denote
\[
P(E)\define \mu(E)=\Pi(E\times \Sigma \,\,, P(F)\define \nu(F)=\Pi(\Theta\times F)
\]
where the second equalities come from lemma \ref{lemma:inteegratemarkovkernelmarginalprobability}. Following this principle we will always denote $\theta \times \Sigma$ simply by $\theta$ whenever the context is clear.
The second step of abuse of notation is motivated by the case where the parameter space $\Theta$ is finite.\\

\begin{lemma}
Assume $\Theta$ and $\Sigma$ are finite spaces and let $\Pi$ be as in \ref{def:bayesianmodel}. Then
\begin{itemize}
\item $\mu_\r{S}(\theta,F)=\Pi(\theta\times \Sigma \vert \Theta\times F)\define P(\theta\vert F)$
\item $P_\r{O}(E,x)=\Pi(E\vert x)$
\end{itemize}
\end{lemma}

\begin{proof}
By definition, we have
\[
\Pi(\theta \times F)=P(\theta\cup F)=\int_{\theta\times \Sigma} \mu_\r{S}(-,F)d\Pi=\mu_\r{S}(-,F)P(\theta)
\]
and vice versa.	
\end{proof}

Hence we will now denote the kernel $\mu_{S}(\theta , F)$ by $P(\theta \vert F)$ and $P_\r{O}(E,x)$ by $P(E\vert x)$

We call $P(E)$ on $\Theta$ and $P(F)$ on $\Sigma$ the prior and predictive probabilities whereas $P(E\vert x)$ and $P(\theta\vert F)$ are the sampling and posterior kernels respectively.

\begin{example}
A good example to keep in mind is the following: assume we have a bunch of emails from people. We let $\Theta$ be the set of people, $\Sigma$ the set of words in the emails and $\Pi($Allen, work) be the number of times Allen has written the word 'work' (normalized). In this case the prior  $P$ is the number of words a person has written, the predictive $P$  is the number of people having written a given word. The sampling kernel $P$(work $\vert$ allen) is the probability among the words written by allen, we pick 'work' and the posterior probability $P$(Allen $\vert$ work ) is the probability that the word work was in fact written by Allen
\end{example}
 
We now assume that the Bayesian model is dominated (meaning that both the posterior and sampling kernel are dominated). 
In this case it defines two estimators. The $MLE$ given by considering the distribution of kernel $P(- \vert \theta)$, written as $f_{x\vert \theta}$ as well as the maximum a posteriori estimate, given by considering
\[
MAP: \Sigma \mor \d{R}: x \fun \argsup_\theta f_{\theta \vert x}(\theta \vert x)
\] 
 
\section{Bayesian ML Learning} 
 
Based off the Bayesian statistical model we described, one defines a Bayesian Machine learning  model in which one parametrizes \emph{data} by hypothesis.
 Let $H\subset \Hom((\Theta,\r{O}),(\Sigma,\r{S}))$ be a finite set of \emph{hypothesis} endowed with the discrete $\sigma$-algebra. We consider  $D\subset \Theta\times \Sigma, \r{O}\times \r{S}$ as a space  of data.

\begin{definition}
A Bayesian ML model is a statistical model on $H\times D$	
\end{definition}


If we assume given a probability $P$ on $H$. Then there is a canonical way to define a Bayesian ML model by letting
\[
\Pi(h\times F)\define \mu(h)\delta_{h,F}
\]
where $\delta_{h,F}$ if the graph of $h$ lies in $F$ and zero else. We can this the noiseless model with prior $P$.

\begin{lemma}
Consider the above model. and let $VS_{H,F}$, the version space of $F$ be the set of all $h \in H$ whose graph lies in $F$. Then
\begin{itemize}
\item the prior is given by $P(h)$
\item the sampling kernel is given by $P(E\vert h)=\Pi(F\vert h)=\delta_{h,F}$
\footnote{recall that we identify $h with h\times D$ and $F$ with $H\times F$ implicitely}
\item The predictive probability is	given by $P(F)= \mu(VS_H,F)$

\item the posterior kernel is given by $\mu_\r{S}(h,D)=\Pi(h\vert D)=\frac{\delta_{h,F}P(h)}{\mu(VS_{H,F})}$
\end{itemize}
\end{lemma}

We will also be interested in Bayesian ML models wich incorporate noise. Here one is not just interested in wanting $\Pi(h\times F)$ to only be nonzero if the graph of $f$ describes the data. Rather, one defines noise on space as a probability and keeps track of that. We are not sure how to formalize this in the the above context forn now.

\section{Bayesian Networks}

\begin{definition}
Let $(\Omega,\r{A},P)$ be a probability space and $(X_i)_{v \in V}: \Omega\mor \Sigma$ a set of random variables.
A Bayesian network is an acyclic	 quiver  $\Qvr$ with vertices $v$.\\
\end{definition}
Given a bayesian network and a node $i$, the parents of $i$ are all node with an arrow coming in to $i$.

\begin{definition}
We say that the random variables $X_v$ follow the Bayesian network $\Qvr$ if
\[
f_{X_1\wedge \ldots \wedge X_n}=\prod_i f_{X_i \vert \wedge X_{j, \textrm{j parent of i}}}\]
\end{definition}
In the case where the quiver consists of a single node $1$ and a single arrow to $2\ldots n$. We say the network is naive, so that
\[
P(A_1,\ldots A_n)=P(A_1)\prod_i P(A_i\vert A_1)
\]

Where we used the definition of the density function of conditioning one random variable wrt to another.
\section{Naive Bayes}

\begin{definition}
We say that a Bayesian model $\Theta \times \Sigma$ 	is naive if $P(-\vert \theta)$ is given by a naieve bayes network.
\end{definition}



\section{Types of Bayesian Models}

Let $P_\theta$ denote a statistical model. Let $\Qvr$ be a Bayesian quiver.

\begin{definition}
We say that a Bayesian model is of $P_\theta$-$\Qvr$ type if there exist random variables $X_i:\Omega	\mor \Sigma$ whose distributions lie in the family $P_theta$ who follow the Bayesian quiver $\Qvr$
\end{definition}



\subsection{A Naive Bernouilli Model}

We consider the problem of classifiying text messages according to people. We let $\Theta$ be the space of people, let $V$, the vocabulary of words. We order these words to obtain $V^n$ and  let $\Sigma=\{0,1\}^n$ (so that a an element represents a text message containing the words corresponding to places with a 1). We assume the sampling probabilities are bernouilli distributed so that for $\theta \in \Theta$ there exist $\pi_theta \in [0,1]^n$ such that for a vector $x \in \Sigma$:
\[
P(x\vert \theta)\define \prod_i (\pi_\theta)_i^{x_i} (1-(\pi_\theta)_i^{(1-x_i)}
\]
We assume $\Theta$ has a prior distribution given by $P(\theta)$

\begin{lemma}
Let $\Pi$ be a finite Bayesian model with Bernouilli sampling distribution
\[
MAP(x)=\argmax_\theta \prod_i P(x_i\vert \theta)^{x_1}(1-P(x_i\vert \theta)^{1-x_i})P(\theta)
\]
\end{lemma}


\begin{proof}
Let $x \in \Sigma$. Since the model is naive we can assume
\[
P(x\cap \theta)=p(\theta)p(x\vert \theta)
\]	
so that it suffices to maximize $p(\theta)p(x\vert \theta)$
Writing the binomial distribution out and taking $\ln$ yields
\[
\delta_{x,\sigma}\ln(\pi_{x,\theta})+(1-\delta_{x,\sigma})\ln(1-\pi_{x,\theta})+p(\theta)
\]
In order to maximize this, we view this as a function of the variable $\pi_{x,\theta}$ and compute the singular points to get
\[
\delta_{x,\sigma}(1-\pi_{x,\theta})-(1-\delta_{x,\sigma})\pi_{x,\theta}=0\iff \delta_{x,\sigma}=\pi_{x,\theta}
\]

\end{proof}



\section{Estimators and Parameters}

Arguably statistics is about the following problem: We are given a measurable function
\[
X:(\Omega, \r{F}, P)\mor (T,\r{T},\tau)
\]
and assume that $\tau$ dominates $P_X$ so that $P_X$ has a density function $f_X: (T,\r{T},\tau) \mor (\d{R},\r{L})$.\\
We assume given a parameter space
\[
(\Theta, \r{G})\times T\mor \d{R}
\]
such that for each $\theta \in \Theta$, $f_\theta$ is a density function. and $f_X=f_\theta$ for some $\theta \in \Theta$ (called the parameter).\\
An \emph{estimator of the random variable $X$} is a sequence of random variables
\[
\hat{\theta}_n:(\Omega,\r{F},P)\mor (\Theta, \r{G})
\] 
which has \emph{desirable properties.}
\begin{center}
The exact meaning of desirable properties explains the ambiguity of statistics. Since statistics is concerned with making conclusions based on a sample size $x_1,\ldots x_n$ of outcomes as opposed to $X$ itself, we shall only concern ourselves with definitions of this type
\end{center}
 
 
\subsection{Consistent Estimators}

\begin{definition}
An estimator $\hat{\theta}_n$ is consistent if it converges in probability to $\theta$ $\hat{\theta_n}\stackrel{P}{\mor} \theta$. I.e. for any choice $\epsilon$
\[
P(\vert \theta_n-\theta\vert <\epsilon)\stackrel{n\to \infty}{\mor} 1
\]
\end{definition}

we have the following alternate characterization of constitency:

\begin{lemma}
The following are equivalent:
\begin{itemize}
\item the estimator $\hat{\theta_n}$ is consistent
\item $E(\theta_n)=\theta$ and $\Var[\theta_n]=0$	
\end{itemize}
\end{lemma}

\begin{proof}
	
\end{proof}
 
 
\section{Maximum Likelihood Estimation}
Let $(\Theta, \r{O})\Longrightarrow (\Omega, \r{F})$ be a statistical model. Taking associated density functions yields
\[
\Theta\times \Omega \mor \d{R}:(\theta,x)\fun f_\theta(x)
\]
Where $F_\theta$ is the density of the distribution $\theta$. We now assume given a set of points $(x_1,\ldots x_n)$.\\ The purpose of maximum likelihood estimation is to pick the parameter $\theta$ such that the \emph{'likelihood'} of the outcomes $x_i$ is maximized. More precisely, we consider the associated statistical model $(\Theta,\r{O})\Longrightarrow (\Omega^n,\r{F}^n)$, with associated density functions
\[
\Theta\times \Omega^n\mor \d{R}:(\theta,x_1\ldots x_n)\fun \prod_i f_\theta(x_i)
\]


Then the maximum likelihood is the function $\hat{\theta}:\Omega^n\mor \Theta$ given by
\[
\hat{\theta}(x_1\ldots x_n)=\max_{\theta\in \Theta} \prod_i f_\theta(x_i)
\]
whenever this maximum exists. For technical reasons, we sometimes compute the maximum  log-likelihood instead
\[
\max_\theta \sum_i \ln f_\theta(x_i)
\]

\begin{example} (a biased coin)
Let $\Omega\define\{H,T\}\mor\{0,1\}$ be the random variable associated to throwing a coin in the air with $P$(heads)$=p$. Assume we repeat the experiment $n$ times to obtain the random variable. And let $X$ be the number of heads. Then $X$ is binomially distributed with success $p$. assume that after performing the experiment $X$, we obtain the number $i$ of heads. which $p$ yields the maximum likelihood?\\
Well, $X$ is binomially distributed so that
\[
\ln f_p(k)=\ln(\binom{n}{k}p^k(1-p)^{n-k})=\ln\binom{n}{k}+k\ln(p)+(n-k)\ln(1-p) 
\]
setting the derivative with respect to $p$ to $0$ yields $p=\frac{n}{k}$ so that given an outcome of $i$ heads, $p=\frac{i}{n}$ yields the highest likelihood
\end{example}

\begin{example}
Another example comes from logistical regression	
\end{example}

\subsection{Gauss' Principle} the techniaue of MLE can be used to motivate the normal (or Gaussian) distribution.
\begin{definition}
we say that a probability on $\d{R}$ is normally distributed if it is dominated by the Lebuesgue measure with density function
\[
\frac{1}{\sqrt{2\pi}}e^{\frac{-x^2}{2}}
\]
Or more generally after the change of variable $x\fun \frac{x-\mu}{\sigma}$
\[
\frac{1}{\sqrt{2\pi}}e^{\frac{-\frac{(x-\mu)}{\sigma}^2}{2}}
\]
\end{definition}
\begin{lemma}
if $X$ is normally distributed, then
\begin{itemize}
\item $X[X]=\mu$
\item $S[X]=\sigma$	
\end{itemize}

\end{lemma}

One interesting characterization of the normal distribution is that it is \begin{center}
the sole distribution whose MLE always coincides with the sample mean\end{center}
To give a more detailed explanation of this sentence, we start with the following lemma:
\begin{lemma}
Let
\[
f:\d{R}\times \d{R}\times \d{R}\mor \d{R}:(\mu,\sigma,x)\fun \frac{1}{\sqrt{2\pi}}e^{\frac{-\frac{(x-\mu)}{\sigma}^2}{2}}
\]	
Then the MLE associated to the sample $x_1\ldots x_n$ is
\[
\hat{\mu}=\sum \frac{x_i}{n}\textrm{ and }\hat{\sigma}=\big(\frac{1}{n}\sum(x_i-\hat{\mu})^2)^\frac{1}{2}
\]
\end{lemma}
\begin{proof}
This is a straightforward exercise.	
\end{proof}
\section{Estimators}

\begin{definition}
Let $X:(\Omega,\r{A},P)\mor (T,\r{T})$ be a random variable. and $(x_1,\ldots , x_n)$ a sample (set of outcomes) An estimator consists of a daigram
\begin{displaymath}
\xymatrix{
&  \Hom(\Omega,\r{A},P), (T,\r{T}))\ar[dr]_\mu\\
\Theta\ar[ur]_h & & \d{R}
}	
\end{displaymath}
such that $\Theta\mor \d{R}$ attains a maximum for a certain function $h_\theta$, the estimation
\end{definition}


There are two important classes of estimation:

\subsection{Maximum likelihood estimation}
 In this case, we suppose given $x_1,\dots x_n$ outcomes of the random variable $X$. Given the map $\Theta \mor \Hom(\Omega,\r{A},P), (T,\r{T}))$, for $h_\theta \in \Theta$, we let \[\mu(h_\theta)=f_{h_\theta^n}(x_1,\ldots x_n)\]
 This number is the likelihood of $(x_1,\ldots x_n)$ assuming $\theta$
 
 \subsection{Maximum A Posteriori Estimate}
 We assume that $\Theta,\r{A},P)$ now has the structure of a probability space and let $ d$ given outcomes $x_1,\ldots x_n$ as well as a random variable $\Theta\mor \d{R}$
\section{on Classifiers}

We start by definining a predictor as follows:
\begin{definition}
For a probability space $\Omega$ and a measurable space $\r{O}$. A cost is a function
\[
C:\Hom(\Omega,\r{O}) \mor \Hom(\Omega,\d{R})
\]	
Let $f\le g \iff \c(f)\le \c(g)$. This defines a pseudo-order. We sat that $f$ is a predictors if its equivalence class $\bar{f}$ is maximal.
\end{definition}

Usually one builds from the data of a training set, which is a map $\tau: \r{T}\mor \r{O}$ where $\r{T}\subset \Omega$

\section{Naive Bayes}

\begin{definition}
We say that a Bayesian model $\Theta \times \Sigma$ 	is naive if $P(-\vert \theta)$ is given by a naieve bayes network.
\end{definition}



\section{the Naive Bayes Classifier}


In this section, we consider the following classification problem:
assume we are given a probability space partitioned into a finite set of outcomes (which we call authors here) $O_k\subset \r{O}$. We are given a set of words $D$ and let $\Omega=D^n$ assume given a feature set $\r{F}\subset \Omega\def \Omega$ (these are usually interpreted as bits of text). We now have a function $\tau:\r{F}\mor \r{O}$ which maps each text to its author. By abuse of notation for an author, we let $P(O_k)=P(\tau^{-1}(O_k))$ and for a word $\omega \in \Omega$, we let $P(\omega)\define P(\cup_i\pi^{-1}_i(\omega))$ (where $\pi_k$ projects a text to its $k$-th word)
Then we define the cost as
\[
\Hom(\Omega,\r{O})\mor \Hom(\Omega,\d{R}):f\mor P(f(\omega)\cap \cap_i \pi_i(\omega))
\]

We say that the problem is naive if the following condition is met for a text $omega \in \Omega$ consisting of words $(\omega_1,\ldots \omega_n)$, we have
\[
P(\omega_1\vert \omega_2\ldots \omega_n)=P(\omega_1)
\] 
Then the predictor associated to this problem is simply given by

\[
NB(\omega)= \max_k P(O_k\cap \pi_1(\omega)\cap \ldots \pi_n(\omega))
\]
It can be easily checked assuming naivity that
\[
NB(\omega)=\max_k \bigg(P(\pi_1(\omega)\vert O_1)\cdot \ldots\cdot P(\pi_n(\omega)\vert O_k)P(O_k)\bigg)
\]

\begin{example}
Assume we have a set of emails from John and Sarah (split 50/50). Those emails contain the words brown fox jumps in the following way: for john we have $(0.1,0.3,0.1)$ and Sarah $(0.2,0.5,0.1)$ We wish to know who wrote the word Brown fox. 
Assume naivity (wish is realistic in this scenario), we have 
$NB_{John}=0,1\cdot 0.3\cdots 0.5$ whereas $NB_{Sarah} =0.2\cdot 0.5\dots 0.5$. So Sarah wrote it. 	
\end{example}

\end{document}