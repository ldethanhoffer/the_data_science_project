%identitification
\NeedsTeXFormat{LaTeX2e}
\ProvidesPackage{preamble}[2020/04 custom preamble package for the data science project]

\begin{document}
\title{PROBABILITY}

\maketitle	


\noindent\hrulefill
\tableofcontents
\noindent\hrulefill

\phantomsection
\label{section-phantom}



\section{Overview}



\section{Probability Spaces}
\label{section:measurable_spaces}

\begin{definition}
\label{definition:measurable_space}
Let $\Omega$ be a set. A $\sigma$-algebra on $\Omega$ is a collection $\r{F}\subset \pr{P}(\Omega)$ of subsets of $\Omega$ such that
\begin{itemize}
\item $\Omega \in \r{F}$
\item $\forall A, B \in\r{F}: A\setminus B \in\r{F}$
\item if $A_i\in\r{F},\,\,\forall \,i \in \d{N}$ then $\bigcup_{i \in \d{N}} A_i \in \r{F}$
\end{itemize}
A measurable space consists of a couple $\big(\Omega,\r{A}\big)$ where $\r{F}$ is a $\sigma$-algebra on $\Omega$
\end{definition}

\begin{definition}
\label{definition:probability_space}
Let $(\Omega,\r{F})$ be a measurable space. A measure on $(\Omega,\r{F})$ is a map $\lambda:\r{A}\mor \d{R}_+ \cup \{\infty\}$ such that 
\begin{itemize}
\item $\lambda\big(\varnothing) = 0$
\item $\lambda\big(\bigcup A_i\big) = \sum_{i \in \d{N}} \lambda\big(A_i\big)$
\end{itemize}
We say that $\lambda$ is a \emph{probability} if $\lambda(\Omega)=1$
We call the resulting triple a measure (resp. probabbility) space.
\end{definition}

\begin{definition}
\label{definition:random_variable}
Let $(U, \r{U}, \lambda)$ be a probability space and $(\Omega, \r{F})$ a measurable space. A random variable is a function
\[
X: U\mor \Omega
\]
such that $X^{-1}(\r{F})\le \r{U}$
\end{definition}

\begin{lemma}
\label{lemma:pushforward}
Let $X: (U, \r{U}, P)\mor (\Omega, \r{F})$ be a random variable. Then
\[
X_*P: \r{F}\mor \d{R}:  A\fun P(X^{-1}(A)) 
\]
defines a probability.
\end{lemma}

\begin{definition}
\label{definition:pushfoward}
The probability $X_*P$ defined above is the \emph{pushforward of $P$ along $X$}.\\
We write $X\sim Q$ if $X_*P =Q$
\end{definition}
\section{a Taxonomy of Probabilities}
\subsection{Discrete probabilities}
\begin{definition}
We say that a probability $P$ on $(\Omega,\r{F})$ is discrete if there exists a sequence $(x_i)_{i \in \d{N}}$ such that
\[
P\big(\Omega\setminus\{(x_i)_{i \in \d{N}}\}\big)=0
\]
\end{definition}
Note that in particular any probability on a countable set is discrete
\subsubsection{Bernouilli}
\begin{definition}
Let $\Omega\define \d{Z}_2$ and $\pr{F}=\pr{P}(\Omega)$. Then any probability on $\Omega$ is called Bernouilli.\\
If $p=P(1)$, then the density is given by
\[
f_P(x) = p^x(1-p)^{1-x}
\]
\end{definition}


\subsection{Continuous Probabilities}



\section{Sampling}



\begin{convention}\label{conv:prob_univ}
We will fix a probability space $(\textrm{U}, \r{U},\lambda)$ and throughout any random variable $X$ on $(\Omega,\r{F})$ will always have this space as its domain.
\end{convention}

\begin{definition}\label{prob:def:sampling}
Let $(\Omega, \r{F})$ be a measurable space.\\
The \emph{sample space} of $\Omega$ is the set 
\[
\sample{\Omega}\define \coprod_{i \in \d{N}}\Omega^i
\]
endowed with the canonical $\sigma$-algebra $\coprod_{i \in \d{N}} \r{F}$.\\
The \emph{drawing space} is the subspace of $\sample{\Omega}$ defined as:
\[
\draw{\Omega} = \big\{ (w_1,\ldots w_n) \in \sample{\Omega} \vert \, \omega_i\neq \omega_j \big\}
\]
again, endowed with the canonical $\sigma$-algebra.\\
The space of random samples is defined as:
\[
RS\Omega\define  \coprod_{i \in \d{N}} \bigg\{(X_1,\ldots X_n)\vert \, X_i \textrm{ i.i.d }\}
\] 
We say that $(x_1\ldots x_n)$ is sampled from $P$ if it lies in the image of the canonical map
\[
\ev: U\times RS\Omega\mor S\Omega: \bigg(u, \big(X_1,\ldots X_n\big)\bigg)\fun \bigg(X_1(u),\ldots X_n(u)\bigg)
\]
and $X_i\sim P$.
Finally, we may need to restrict the choice of probabilities in the future. To this end, assume $\big(P_\theta\big)_{\theta \in \Theta}$ is a collection of probabilities on $(\Omega,\r{F})$. Then we define
\[
\rsample_\theta \Omega \define \bigg\{( X_1,\ldots X_n) \in \rsample\Omega\,\,\vert \,\, \exists \theta: X_i\sim P_\theta \bigg\}
\]
\end{definition}


\begin{remark}\label{rem:prob_sampling-map}
The above map $\ev: \Omega\times RS\Omega\mor S\Omega: $
will be referred to as the sampling map
\end{remark}

\end{document}